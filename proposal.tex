\title{Proposal}

\documentclass[12pt]{article}

\begin{document}
\maketitle

\section{Abstract}
%Age of big data and explosion of applications
%new ways to develop applications, event-driven and lambdas.
%Sandboxing therefore needs to adapt to that new paradigm.
%per function/lambda for isolated stateless routines.
%Cheap isolation cost equal to switch of cr3?

%trial 1
%Mobile and web applications followed an exponential growth over the past decade.
%The play store, for example, went from 16 000 applications in 2009, to nearly 2 600 000 in 2016.
%As a new privileged way to reach customers.

%mobile platforms -> user interaction.
%Success overnight, require scaling etc. so cloud services and hosting solutions blabla.
%Also the fact that they have to run untrusted code.

%trial 2
%Web service providers had to adapt to the growing number of web applications during the past decade.
%In order to stay competitive, they had to provide new abstractions to their customer by exposing simple %abstractions that allow flexible development, integration, and deployment of mobile and web applications backend %logic.

%trial 3
%Web services providers strive to provide flexible ways to develop, deploy, and maintain mobile and web %applications.
%As these applications have, nowadays, heterogeneous sources, often.

%trial 4
%Cloud services integration of new solution that allow application developers to focus on the app's logic, rather %than deployment etc.%
%out-of-the-box.%
%The logic is broken down to executing stateless routines in response to defined events.
%As a result, such services needs to run untrusted code etc.

%Cloud services provide application developers with frameworks and eco-systems that take care of server %management, application deployment, and guarantees such as automatic scaling of computational resources% and %availability.%
%Hence, application developers only need to implement routines that handle application's events and requ%ests.
%Such routines are stateless functions that are executed on the server side whenever the application triggers an %event.
\subsection{version 1}
The apparition of the recent cloud-based model to develop mobile and web applications requires to re-evaluate the granularity of existing sandboxing mechanisms.
%Cloud services provide ecosystems in which user inputs and requests generate pre-defined events.
Developers can execute stand-alone functions that run on the backend servers and handle requests, while the cloud service provider takes care of server management, resource allocation, and runtime environments.

While very attractive to application developers, this function-based model presents a new security and availability challenges for cloud service providers.
Arbitrary and untrusted code needs to be executed on servers, while enforcing isolation among co-located applications and preserving the servers integrity.
Common sandboxing techniques, such as virtual machines and containers, are, however, inadequate to this per-function isolation model.

This paper presents NAME, a function sandboxing mechanism that relies on hardware features to ensure isolation between untrusted user-defined functions.
Our solution leverages hardware support for virtualization to efficiently generate virtual address spaces for each stateless function.
We claim that our solution presents a smaller overhead, both in terms of execution and resource consumption, than standard sandboxing mechanisms for per-function isolation.\\


\subsection{version 2}
%Existing sandboxing mechanisms need to be reevaluated in order to better fit the recent models used to develop mobile and web applications.
%Giant tech companies, such as Google and Amazon, compete to provide out-of-the-box solutions for application developers that generate a new development model, i.e., developers are responsible for implementing the application's logic as a set of functions, and do not have to consider operational apsects such as server deployment, resource management, and runtime environments.

Sandboxing mechanisms need to be re-evaluated to better fit the emergence of new programming models for mobile and web applications.
Nowadays, application developers outsource operational apsects, such as server deployment, resource management, and runtime environments by relying on out-of-the-box solutions.
Such OOTBs handle function rather than machine abstractions and hence require isolation between co-located applications and to preserve the host's integrity.

However, common sandboxing techniques, such as virtual machines or containers, are inadequate to provide a per-function isolation mechanism.
A new abstraction, with smaller overheads, is therefore required to isolate functions from each other and prevent them from hurting the underlying host.

This paper presents NAME, a function sandboxing mechanism that relies on hardware features to ensure isolation between untrusted user-defined functions.
Our solution leverages hardware support for virtualization to efficiently generate virtual address spaces for each stateless function.
We claim that our solution presents a smaller overhead, both in terms of execution and resource consumption, than standard sandboxing mechanisms for per-function isolation.





\end{document}
