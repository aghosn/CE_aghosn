%General
%Protection of access.
%CPU keeps metadata about the pages (put EPC and SECS here and CPU enclave mode)
% SGX attestation.
% 
% Only trusted part is the CPU itself. 
The Intel Software Guard Extension (SGX) defines a set of new CPU instructions that allows to create and manipulate \emph{enclaves}.
An enclave is a portion of the user address space protected against unauthorized access or modification by software, regardless of its privilege level.
SGX ensures data confidentiality, integrity, and prevents replay attacks on the enclave's content.
In this scheme, the processor acts as the root of trust to achieve the protected execution of enclaved programs.

Enclave's code and data become available once the processor \emph{enters} the enclave.
To enter an enclave, the processor switches to \emph{enclave mode}, loads the appropriate page mappings, and jumps to a specific entry point contained within the enclave.
While cache resident data is protected by the translation lookaside buffer (TLB), SGX relies on cryptographic techniques to ensure the confidentiality and integrity of an enclave's page in main memory.

Main memory, i.e., DRAM, is not part of the trusted computing base (TCB).
As a result, in enclave mode, all traffic between the CPU and main memory needs to be encrypted and integrity protected.
SGX relies on a dedicated hardware unit, called \emph{Memory Encryption Engine} (MEE), that is part of the TCB and operates as an extension of the Memory Controller (MC).
The MEE is in charge of a portion of the physical memory, i.e., the \emph{protected region}, determined at boot time, that holds enclave pages, the processor's metadata used to handle them, and the region's integrity tree.
The MEE implements and enforces the encryption and integrity protection schemes for enclave resident data.
More specifically, it decrypts and verifies the integrity of enclaves data loaded from main memory, and encrypts and computes MAC tags and version attributes to protect the integrity of (and prevent replay attacks on) data written to main memory.
A failure to verify an enclave's data integrity results in a fault that blocks the processor's MC and requires a machine reboot.

The processor stores enclave pages and their metadata in the \emph{Enclave Page Cache} (EPC), a subrange of the protected region.
Each enclave is associated with a corresponding \emph{SGX Enclave Control Structure} (SECS) that resides within the EPC and is used by the hardware to validate interactions with enclave pages.
The SECS tracks physical pages that belong to a particular enclave, their types, their corresponding virtual mappings, as well as their access rights.
Each enclave might contain several thread control structures (TCS), that encapsulate the state and entry point for threads executing in the enclave.
TCSs belong to exactly one SECS and are also stored in the EPC.
Finally, the EPC also acts as a cache holding free enclave pages and allows to dynamically extend the memory allocated to a specific enclave.

SGX implements local and remote attestation mechanisms to verify the integrity of a newly created enclave.
Enclaves are instantiated by untrusted software.
As a result, an extra verification step is required to prove that both software and hardware environments were initialized as expected.
Local attestation enables one \emph{reporting} enclave to prove its genuineness to a \emph{target} enclave via a symmetric key cryptographic scheme.
The reporting enclave triggers a procedure taking as argument the target's enclave identity.
The procedure generates a \emph{report} containing a MAC of itself signed by the processor with the target's secret key.
A report encapsulates both hardware and software environments, referred to as the \emph{enclave's identity}.
The report is then sent to the target enclave that can verify its authenticity by recomputing the signed hash and comparing it with the one provided.

Remote attestation relies on asymmetric cryptography to produce a report.
First, the local attestation is used with a special \emph{quoting enclave}.
The quoting enclave then generates and inserts a MAC in the report computed with a private, device specific key and sends the report to the remote entity.
The known public key can at that time be used to verify the authenticity of the MAC contained within the report.
A remote attestation report is called a \emph{quote} to differentiate it from local attestation reports.

The next section presents Haven's design and how the authors leveraged Intel SGX to provide shielded execution of unmodified legacy applications.