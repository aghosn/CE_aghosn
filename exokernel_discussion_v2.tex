Although originally published two decades ago, the Exokernel addresses modern concerns in terms of system design and application performance.
It allows application-level management of resources by exposing an interface as close as possible to the underlying hardware, hence removing the need for kernel bypass in order to increase performance \cite{BelayPKGKB14,DBLP:journals/tocs/CaoFKL96}.
The Exokernel provides a small set of primitives to securely expose and multiplex hardware resources.
This reduced set of responsibilities allows a compact and simple kernel implementation that is less prone to implementation bugs, exposes a smaller attack surface, and is highly efficient.

The Exokernel exposes an interface close to what a Cloud client actually rents from the service provider, i.e., raw hardware resources.
Such a design is therefore also relevant with regards to modern application deployments. 

Furthermore, the Exokernel provides a clean separation between the kernel (host) and the application (guest).
We consider this as a necessary first step towards bi-directional isolation.

Finally, while the Exokernel architecture is a potential base for a modern system design, we note that its essential mechanisms, i.e., secure bindings, are highly dependent on the underlying hardware features.
As a result, implementing an Exokernel requires to either limit the target hardware setup, e.g., provide support and device drivers for only a specific set of hardware configurations, or ensure that the kernel can cope with heterogeneous environments.
The second option seems unpractical and would result in a complicated system.

With the next paper, we study how the Exokernel high-level design, coupled with virtualization, alleviates the need for heterogeneous hardware support and enables the aggressive specialization of the application stack.
We further examine how such a solution integrates with Cloud deployments.



