\section{Shielding Applications from an Untrusted Cloud with Haven}

Cloud computing service providers enable even small organizations to deploy web-based services quickly, with low start-up costs, and efficiently adapt the amount of available resources to their current load.
On a machine, the cloud service provider's host divides physical resources among co-located applications from different origins.
While attractive for their simplicity and adaptability, such services raise important confidentiality and integrity concerns that limit their adoption \cite{mining,NSAYahoo}.
Cloud services implement a unidirectional isolation model where privileged code (the host) is protected from the unprivileged one (the guest), and retains unrestricted access to the application's data.
In other words, cloud users are forced to trust the entire cloud privileged management software stack with their sensitive data and intellectual property.
As privacy and confidentiality are growing concerns, new techniques need to be devised to protect the application's data from a potentially compromised or malicious host.

Haven \cite{DBLP:journals/tocs/BaumannPH15} is a system that leverages Intel Software Guard Extension (SGX) \cite{SGXManual} to provide shielded execution of unmodified legacy applications on a commodity OS (Windows).
Shielded execution allows to protect the \emph{confidentiality} and \emph{integrity} of a program from the host (e.g., the OS) on which it runs.
Confidentiality ensures that the program's intermediate states are not observable by the rest of the system, while integrity guarantees that if the program completes, the output is correct.
We briefly introduce the Intel SGX extension main features before presenting Haven's design.

\subsection{The SGX extension}

%General
%Protection of access.
%CPU keeps metadata about the pages (put EPC and SECS here and CPU enclave mode)
% SGX attestation.
% 
% Only trusted part is the CPU itself. 
The Intel Software Guard Extension (SGX) defines a set of new CPU instructions that allows to create and manipulate \emph{enclaves}.
An enclave is a portion of the user address space protected against unauthorized access or modification by software, regardless of its privilege level.
SGX ensures data confidentiality, integrity, and prevents replay attacks on the enclave's content.
In this scheme, the processor acts as the root of trust to achieve the protected execution of enclaved programs.

Enclave's code and data become available once the processor \emph{enters} the enclave.
Entering an enclave requires the processor to switch to \emph{enclave mode}, load the appropriate page mappings, and jump to a specific entry point contained within the enclave.
While cache resident data is protected by the table lookaside buffer (TLB), SGX relies on cryptographic techniques to ensure the confidentiality and integrity of an enclave's page in main memory.

Main memory, i.e., DRAM, is not part of the trusted computing base (TCB).
As a result, in enclave mode, all traffic between the CPU and main memory needs to be encrypted and integrity protected.
SGX relies on a dedicated hardware unit, called \emph{Memory Encryption Engine} (MEE), that is part of the TCB and operates as an extension of the Memory Controller (MC).
The MEE is in charge of a portion of the physical memory, i.e., the \emph{protected region}, determined at boot time, that holds enclave pages, the processor's metadata used to handle them, and the region's integrity tree.
The MEE implements and enforces the encryption and integrity protection schemes for enclave resident data.
More specifically, it decrypts and verifies the integrity of enclaves data loaded from main memory, and encrypts and computes MAC tags and version attributes to protect the integrity of (and prevent replay attacks on) data written to main memory.
A failure to verify an enclave's data integrity results in a fault that blocks the processor's MC and requires a machine reboot.

The processor stores enclave pages and their metadata in the \emph{Enclave Page Cache} (EPC), a subrange of the protected region.
Each enclave is associated with a corresponding \emph{SGX Enclave Control Structure} (SECS) that resides within the EPC and is used by the hardware to validate interactions with enclave pages.
The SECS tracks physical pages that belong to a particular enclave, their types, their corresponding virtual mappings, as well as their access rights.
The EPC also acts as a cache holding free enclave pages and allows to dynamically extend the memory allocated to a specific enclave.

SGX implements local and remote attestation mechanisms to verify the integrity of a newly created enclave.
Enclaves are instantiated by untrusted software.
As a result, an extra verification step is required to prove that both software and hardware environments were initialized as expected.
Local attestation enables one \emph{reporting} enclave to prove its genuineness to a \emph{target} enclave via a symmetric key cryptographic scheme.
The reporting enclave triggers a procedure taking as argument the target's enclave identity.
The procedure generates a \emph{report} containing a MAC of itself signed by the processor with the target's secret key.
A report encapsulates both hardware and software environments, referred to as the \emph{enclave's identity}.
The report is then sent to the target enclave that can verify its authenticity by recomputing the signed hash and comparing it with the one provided.

Remote attestation relies on asymmetric cryptography to produce a report.
First, the local attestation is used with a special \emph{quoting enclave}.
The quoting enclave then generates and inserts a MAC in the report computed with a private, device specific key and sends the report to the remote entity.
The known public key can then be used to verify the authenticity of the MAC contained within the report.

\subsection{Design}

\begin{figure*}
	\centering
	\captionsetup{justification=centering}
  	\includegraphics[scale=0.3]{haven}
  	\caption{Haven components and interfaces from \cite{DBLP:journals/tocs/BaumannPH15}}
  	\label{stackOfHaven}
\end{figure*}

Haven leverages the low level isolation mechanisms provided by SGX to achieve shielded execution of legacy application binaries.
Since the host OS cannot be trusted, Haven provides most of the operating system services required by the application and manages policies and mechanisms for virtual resources, while the host retains management over physical ones.
This clear separation, coupled with defensive programming and careful interactions between guest and host, is supposed to facilitate Haven's resistance to so-called Iago attacks \cite{DBLP:conf/asplos/CheckowayS13}, i.e., a malicious host exploiting the application's trust to achieve its goals.

Haven's definition of shielded execution requires to
\begin{enumerate*}
	\item bootstrap and validate the execution environment within the enclave,
	\item implement virtual resource management within the enclave, and
	\item defend against a misbehaving host by verifying the result of every service requested from it%, and
	%\item prevent the host from gathering information on the enclave's state via exception handlers
\end{enumerate*}.
We now describe how Haven's design addresses each of these concerns.
Figure \ref{stackOfHaven} provides an overview of Haven's runtime stack.
The application, the LibOS, and the shield module form the trusted software base.
They are all encapsulated inside the enclave.

Haven wraps the application in a sandbox provided by Drawbridge \cite{DBLP:conf/asplos/PorterBHOH11}.
The application executes within a secured container, called \emph{picoprocess}, that exposes a narrow ABI to request OS services.
As a standard sandboxing mechanism, the picoprocess protects the host from the guest.

The application relies on a trusted library operating system (LibOS) to request system services.
In Haven, the LibOS is a modified version of Windows 8, retrofitted into a set of libraries, that implements the full OS API.
The LibOS is under user control and can be inspected and verified by the application.

In order to protect the application from Iago attacks, Haven relies on a \emph{shield module}.
The shield module exposes the drawbridge ABI to the LibOS and implements kernel functionalities and primitives such as memory management and thread synchronization.
It isolates the LibOS from a malicious host by validating all parameters and results that cross the border between trusted and untrusted parts of the runtime.
Such validations depend on the specific service required from the host OS, but mostly consist in sanity checks and observably consistent results.

The untrusted interface at the boundary of the enclave is designed to allow correctness verification by the shield, while being flexible enough to permit an efficient implementation of services on both sides.
The host manages physical resources while the guest handles virtual ones.

Underneath the untrusted interface, an untrusted runtime module forwards calls between the guest and the host.
This module also acts as a bootstrap to create the enclave and load the shield.

The host OS interacts with the enclave via the Drawbridge ABI and provides services relating to memory management, such as, changing memory access rights or allocating untrusted virtual memory, thread signaling, and access to untrusted external devices and to the system time.

Applications are deployed in the cloud as encrypted disk images that contain both the application and the LibOS binaries.
The cloud service provider is responsible for creating the picoprocess and loading the untrusted runtime, which in turn creates the enclave and loads the shield module.
SGX remote attestation mechanism is used to compute a quote of the shield and verify its integrity as well as the enclave's environment.
Subsequently, the shield generates asymmetric cryptography keys, securely sends its public key to a machine controlled by the user along with the quote.
The user verifies the shield's integrity, encrypts the disk image with the public key, and sends it to the shield.
The application is then ready to be loaded.

Haven stacks several software layers, each providing specific functionalities to the ones above it.
This layered architecture, coupled with the defensive programming design adopted, introduces non-negligible overheads.
The next section focuses on Haven's reported performance.


\subsection{Evaluation}

The authors were provided with a functional SGX emulator to implement Haven.
Unfortunately, such an emulator cannot be relied on to measure, or even approximate, the system performance.
As a result, a second version of Haven that introduces synthetic delays for SGX instructions was used for instead.
The performance evaluation hence consists in two steps:
\begin{enumerate*}
	\item compare the performance of Haven with other deployments, where SGX instructions have a null cost and enclave crossings only incur a TLB flush, and
	\item vary the synthetic delays for SGX main features, i.e., enclave crossings, dynamic memory allocation, and accesses to the EPC
\end{enumerate*}.

The first step of the evaluation allows to measure overheads that are purely due to Haven's implementation.
The second step models the performance impact of the SGX hardware.
By cross-referencing Haven's reported results with a recent performance evaluation of SGX\cite{DBLP:conf/IEEEwisa/ZhaoSTZX16}, we are able to get a better understanding of Haven's actual performance.

The first evaluation relies on two real-world applications, namely, a TPC-E benchmark running on a Microsoft SQL server, and an Apache HTTP server benchmark.
Haven's performance is compared with the throughputs achieved by running the same experiments natively, in a hyper-V VM, and in Drawbridge.
As a general result, Haven's throughput is inferior to all other configuration for both applications.
On the database benchmark, it incurs a 13\% slowdown compared to Drawbridge.
On the HTTP server, the same comparison yields a 40\% slowdown.
We note however that Haven running with a dedicated filesystem outperforms Drawbridge, due to the read-intensive nature of the workload that access cached pages inside the enclave.

The second step of the evaluation introduces synthetic delays for enclave crossings, dynamic memory allocation, and accesses to the EPC.
Varying the access latency for EPC pages is done by changing the DRAM clock rate.
Slowing down the memory by a third introduced a 21\% slowdown for the database application, and a 7\% one for the HTTP server.
According to the SGX performance evaluation in \cite{DBLP:conf/IEEEwisa/ZhaoSTZX16}, memory operations inside the enclave are more than three times slower than outside of it.
It therefore seems that Haven's authors largely underestimated the cost of accessing the EPC.

The authors note that while enclave crossings and dynamic allocation of memory both have impacts on the HTTP server application's throughput, the database application seems to be sensitive only to enclave crossings.
By plugging-in the SGX performance evaluation result from \cite{DBLP:conf/IEEEwisa/ZhaoSTZX16}, i.e., 7 kcycles per crossing, we see that enclave crossings induce a 20\% extra slowdown in throughput in the previously measured Haven's database experiment.
For the HTTP server, the actual slowdown is of the order of 1\% for enclave crossings.

More generally, Haven's authors predicted a 31\% to 54\% slowdown in performance compared to a virtual machine solution.
As mentioned above, memory operations on EPC pages were largely underestimated.
A thorough experiment, on real hardware, is needed to better understand how Haven really compares to VM based solutions.
We also note that, while targeting Cloud services, Haven does not provide the same flexibility as VMs in terms of deployment and migration of applications.


\subsection{Discussion}

Haven addresses modern confidentiality and integrity concerns that sprung from the growth of Cloud services.
At its heart, Haven relies on the same fundamental design principle as the exokernel, i.e., it separates policies and mechanisms.
Here, this design choice enables to disentangle the guest from the host, obtain a cleaner separation between mutually distrustful domains, and more easily monitor their interactions.
We note, however, that Haven still relies on the host to access specific services, such as storage or network devices.

We believe that the exokernel architecture provides a stricter separation between a host and a guest that would allow a better implementation of shielded execution.
In the exokernel, the interactions between host and guest are even more limited than in Haven.
The host and guest only need to cooperate to establish secure bindings for a specific resource.
This alleviates the need to rely on an untrusted host to access services such as, for example, external storage.
Furthermore, the exokernel provides mechanisms to cope with the scarcity of enclave pages.
The number of enclave pages is fixed at boot time and represents only a portion of the available main memory.
The exokernel revocation protocol, that allows application-level management of revoked resources, could therefore be leveraged to alleviate the system's physical limitations.
Finally, the abort protocol for misbehaving applications would require a careful implementation to avoid triggering an SGX fault and blocking the MC, but seems feasible.

Haven adopts a radical position by providing support for legacy applications.
This decision was probably made to test SGX flexibility and limits.
In practice, backward compatibility could be sacrificed as in the unikernel paper.

Unikernels present several advantages regarding shielded execution and improve upon Haven's limitations.
First, a Mirage appliance can seal its memory mappings.
This mechanism avoids dynamic allocation of memory resources and hence removes the associated overheads reported in Haven.
Second, Mirage appliances are compact binaries.
Haven packs an entire Windows operating system inside the enclave, which increases the enclave pages consumption and cache misses.
The small size of Mirage unikernels could potentially solve both issues and exhibit better runtime performances.









