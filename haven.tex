\section{Shielding Applications from an Untrusted Cloud with Haven}

Cloud computing service providers enable even small organizations to deploy web-based services quickly, with low start-up costs, and efficiently adapt the amount of available resources to their current load.
On a machine, the cloud service provider's host divides physical resources among co-located applications from different origins.

While attractive for their simplicity and adaptability, such services raise important confidentiality and integrity concerns that limit their adoption[CITE FROM HAVEN].
Cloud services implement a unidirectional isolation model where privileged code (the host) is protected from the unprivileged one (the guest), and retains unrestricted access to the application's data.
In other words, Cloud users are forced to entrust the entire Cloud privileged management software stack with their sensitive data and intellectual property.
Seeing as privacy and confidentiality are growing concerns, new techniques need to be devised to protect the application's data from a potentially compromised or malicious host.

Haven[CITE] is a system that leverages Intel Software Guard Extension (SGX) to provide shielded execution of unmodified legacy applications on a commodity OS (windows).
Shielded execution allows to protect the \emph{confidentiality} and \emph{integrity} of a program from the host (e.g., the OS) on which it runs.
Confidentiality ensures that the program's intermediate states are not observable by the rest of the system, while integrity guarantees that if the program completes, the output is correct.
We briefly introduce the Intel SGX extension main features before presenting Haven's design.

\subsection{The SGX extension}

%General
%Protection of access.
%CPU keeps metadata about the pages (put EPC and SECS here and CPU enclave mode)
% SGX attestation.
% 
% Only trusted part is the CPU itself. 
The Intel Software Guard Extension (SGX) defines a set of new CPU instructions that allows to create and manipulate \emph{enclaves}.
An enclave is a portion of the user address space protected against unauthorized access or modification by software, even the ones running at a higher privileged level.
Data within an enclave is encrypted and integrity protected.
\adrien{Only trust the CPU internals.}
Any attempt to modify an enclave's data from outside of it results in a fault.

To read and modify pages that belong to an enclave, the CPU switches to \emph{enclave mode}, i.e., it \emph{enters} the enclave.
While in enclave mode, traffic between main memory and the processor passes through a \emph{Memory Encryption Engine} responsible for decrypting and verifying the integrity of the data loaded from memory, and encrypting and integrity protecting DRAM data that belongs to an enclave.

\adrien{
	That's for the traffic between cache and DRAM.\\

	Cryptographic measurement, sha-256 digest.
	Replay resistent as well!!!

	So the way it works: MEE is part of Memory controller. At boot, it generates random keys, and seizes a region of RAM to store integrity tree. Integrity tree is just a MAC tag of nodes (i.e., pages). Merkel Tree, where only the root is stored in trusted part of the system. Enables to verify everything since has hash of its children.
	Counters for versioning (and avoiding replay attacks.)
}

\adrien{
	Hardware based Attestation.
	Enclave's measurement is a secured record of the content of the enclave's and how it was loaded.
	Attestation of hardware environment and the enclave.
	Enclave has persistent hardware based encryption key.
	user owned special persistent key. Also able to change this.

	Attestation: 1) check code and data, and 2) authority above the encalve.
	1) is basically hash of build log (content, relative position of pages, security flags in pages).
	2) Sealing authority + product ID + version number.

	Signed certificated with expected 1) and public key.
	If everything works, store the public key inside 2).

	Two attestation mechanisms. One local (between two enclaves) and a remote one.
	Local attestation: REPORT (two identities, attributes, hardware TCB, MAC tag).
	MAC is signed with REPORT KEY. Enclave specific. Relies on the MRENCLAVE being known and passed as argument to EREPORT.

	Remote needs asymmetric keys.
	Quoting enclave helps with this process.
	Then device specific (private) key signes the MAC -> outputs a Quote.
	Intel Enhanced privacy ID.
	EPID bound to processor firmware (so Quote is signed by the processor).

	Sealing allows to save the enclave to "replay" it later on.

}



\subsection{Design}

Haven leverages the low level isolation mechanisms provided by SGX to achieve shielded execution of legacy application binaries.
Since the host OS cannot be trusted, Haven provides most of the operating system services required by the application and manages policies and mechanisms for virtual resources, while the host retains management over physical ones.
This clear separation, coupled with defensive programming and careful interactions between guest and host, is supposed to facilitate Haven's resistance to Iago attacks, i.e., a malicious host exploiting the application's trust to achieve its goals.

Haven's definition of shielded execution requires to
\begin{enumerate*}
	\item bootstrap and validate the execution environment within the enclave,
	\item implement virtual resource management within the enclave,
	\item defend against a misbehaving host by verifying the result of every service requested from it, and
	\item prevent the host from gathering information on the enclave's state via exception handlers
\end{enumerate*}.
We now describe how Haven's design addresses each of these concerns.
Figure [REF] provides an overview of Haven's runtime stack.
The application, the LibOS, and the shield module form the trusted software base.
They are all encapsulated inside the enclave.

Haven wraps the application in a sandbox provided by Drawbridge[CITE].
The application executes within a secured container, called \emph{picoprocess}, that exposes a narrow ABI to request OS services.
As a standard sandboxing mechanism, the picoprocess protects the host from the guest.

The application relies on a trusted library operating system (LibOS) to request system services.
In Haven, the LibOS is a modified version of Windows 8, retrofitted into a set of libraries, that implements the full OS API.
The LibOS is under user control and can be inspected and verified by the application.

In order to protect the application from Iago attacks, Haven relies on a \emph{shield module}.
The shield module exposes the drawbridge ABI to the LibOS and implements kernel functionalities and primitives such as memory management and thread synchronization.
It isolates the LibOS from a malicious host by validating all parameters and results that cross the border between trusted and untrusted parts of the runtime.
Such validations depend on the specific service required from the host OS, but mostly consist in sanity checks and observably consistent results.

The untrusted interface at the boundary of the enclave was designed to allow correctness verification by the shield, while being flexible enough to permit an efficient implementation of services on both sides.
In order to achieve both security and performance, Haven's authors decided to decouple policies from mechanisms.
The host is responsible for managing physical resources while the guest handles virtual ones.

Underneath the untrusted interface, an untrusted runtime module exists to forward calls between the guest and the host.
This module also acts as a bootstrap to create the enclave and load the shield.

The host OS interacts with the enclave via the Drawbridge ABI and provides services relating to memory management, e.g., changing memory access rights or allocating untrusted virtual memory, thread signaling, access to external untrusted devices, e.g., storage, and to the system time.

Applications are deployed on the Cloud as encrypted disk images that contain both the application and LibOS binaries.
The Cloud service provider is responsible for creating the picoprocess and loading the untrusted runtime, which in turns creates the enclave and loads the shield module.
SGX remote attestation mechanism is used to compute a quote of the shield and verify its integrity as well as the enclave's environment.
Afterwards, the shield generates asymmetric cryptography keys, securely sends its public key to a machine controlled by the user along with the quote.
The user verifies the shield's integrity, encrypts the disk image with the public key, and sends it to the shield.
The application is then ready to be loaded.

Exception handling requires the host's intervention and must therefore be done with care to avoid leaking information about the enclave's state.
When an exception occurs in enclave mode, SGX saves the content of the registers and information about the fault within the current TCS.
A synthetic context for the exception is then created to prevent data leaks to the host.
This synthetic context is passed to the host OS which can either handle the exception, or report it back to the guest.
In the latter case, the shield verifies the legitimacy of the exception before reporting it to the LibOS.

Haven stacks several software layers, each existing to provide specific functionalities to the ones above it.
This layered architecture, coupled with the defensive programming design adopted, might introduce non-negligible overheads.
The next section focuses on Haven's reported performance.

%The  requirements for Haven imply that the application must be able to verify and trust its entire software stack that is part of the enclave, that services still provided by the underlying host must be treated as potentially misbehaving,  
%Drawbridge running on top of host kernel is responsible for the thing. 
%%Insert image.
%
% Rely on drawbridge to sandbox the application and have a trusted LibOS (trusted by the app itself).
% Exposes a narrower interface, and is verifiable by application. 
% Reducing interface means less ways to exploit and easier to control.
% 
% Shield module within the TCB. Gives core OS operations
% Also a bootloader for libos and application.
% There to protect against Iago attacks.
% 
% Untrusted interface (not in enclave)
% Policy mechanism separation
% Guest is responsible for virtual, host for physical
% Untrusted runtime
% create enclave, loading shield, forwarding calls to host
% Decided to prevent app from allocating anything outside of the enclave.
% Storage?
% 
% Threads and synchronization
% user level scheduling
% 
% Deployement is weird because also need to verify the VHD.
% Basically load and attest shield, then send the disk image.
% 
% Switching in and out of enclave requires specific mechanisms (clearing registers etc.)
% Requires copy for parameters passed by reference
% 
% Exception handler not really interesting.
% 
% 
% Requires to load all binaries directly because no demand loading possible.
\subsection{Evaluation}

The authors were provided with a functional SGX emulator to implement Haven.
Unfortunately, such an emulator cannot be relied on to measure, or even approximate, the system performance.
As a result, the authors implemented a second version of Haven that introduces synthetic delays for SGX instructions.
The performance evaluation hence consists in two steps:
\begin{enumerate*}
	\item Compare the performance of Haven with other deployments, where SGX instructions have a null cost and enclave crossings only incur a TLB flush, and
	\item vary the synthetic delays for SGX main features, i.e., enclave crossings, dynamic memory allocation, and accesses to the EPC
\end{enumerate*}.

The first step of the evaluation allows to measure overheads that are purely due to Haven's implementation.
The second step models the performance impact of the SGX hardware.
By cross-referencing Haven's reported results with a recent performance evaluation of SGX[CITE], we are able to get a better understanding of Haven actual performance.

The first evaluation relies on two real-world applications, namely, TPC-E benchmark running on a Microsoft SQL server, and an Apache HTTP server benchmark.
Haven's performance is compared with the throughputs achieved by running the same experiments natively, in a hyper-V VM, and in Drawbridge.
As a general result, Haven's throughput is inferior to all other configuration in both applications.
On the database benchmark, it incurs a 13\% slowdown compared to Drawbridge.
On the HTTP server, the same comparison yields a 40\% slowdown.
We note however that Haven running with a dedicated filesystem outperforms Drawbridge, due to the read-intensive nature of the workload that access cached pages inside the enclave.

The second step of the evaluation introduces synthetic delays for enclave crossings, dynamic memory allocation, and accesses to the EPC.
First, varying the delay to access EPC pages is done by changing the DRAM clock rate.
Slowing down the memory by a third introduced a 21\% slowdown for the database application, and a 7\% one for the HTTP server.
Second, the authors note that while enclave crossings and dynamic allocation of memory both have impacts on the HTTP server application's throughput, the database application seems to be sensitive only to enclave crossings.
By plugging-in the SGX performance evaluation result from[CITE], we see that enclave crossings induce a 20\% extra slowdown in throughput in the previously measured Haven's database experiment (8 kcycles per crossing).
For the HTTP server, the actual slowdown is of the order of 1\% for crossings.

More generally, Haven's author predicted a 31\% to 54\% slowdown in performance compared to a virtual machine solution.
This estimation seems conservatively correct when actual SGX performances are considered.

To get a better understanding of SGX performance, we briefly report the result from[CITE].
Cross enclave calls require 7 kcycles, which is an order of magnitude greater than a standard system call in a Linux operating system.
Memory operations inside an enclave are a little more than three times slower than outside of it, while cross enclave memory copy exhibits a 50\% slowdown in throughput.

\subsection{Discussion}
%Untrusted time
%VM exits.
%Cannot move as easily as a VM.
%Performance.
%Monitoring is hard when needed.

Still untrusted parts like the time.
Vm exits are a problem.
Cannot move it as easily as a VM, which is a problem since they target cloud deployment.
The performance penalty is pretty high, it improves security but degrades application performance significantly.
Cannot monitor, when on cloud services this is actually required and very important.\\

What we can take away from this paper is actually that applications deployment changed significantly.
Before, we could deploy on our own trusted servers.
Now multiple mutually untrusted applications are co-located on the host and we have this new bidirectional lack of trust between host and guest.
We need to address this issue.
Relying on hardware is an interesting solution because, compared to software, it is less amenable to bugs and hard to attack or circumvent.
The whole libOS thing is also interesting because it shows that by reimplementing the abstractions that regular applications rely on, we can run unmodified applications.
Also allows the application to verify most of the execution stack (e.g., inspect the libOS).
Although authors claim that this design can be easily ported, I see the lack of support for fork as a problem.
Why untrusted channel when SGX has a mechanism to attest a program?

\adrien{Key insight + how to link to exokernel: separation between physical resources and virtual ones, and separation between policies and mechanisms.}




