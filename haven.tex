\section{Shielding Applications from an Untrusted Cloud with Haven}

\subsection{Overview}
%Define shieldied application.
%Give general goal and design.

%Shielded application.
%New problem is that you colocate applications in cloud services and want to be protected against a potentially corrupted host.
%Sandbox usually act the other way around.
%Rely on hardware mechanisms to do so by using the Intel SGX extension.\\
%
%Whole system is based on the libOS drawbridge.\\
%
%\adrien{Redacted}\\
%Haven[CITE] defines the notion of \emph{shielded execution}, i.e., protecting a program's data confidentiality and integrity from the host on which it runs.
%With the emergence of cloud services, Haven's authors identified TODO.
%While conveniently allowing a fast deployment of web-services, Cloud service providers often require their %clients to entrust them with unlimited access to their app-specific data.
%The privileged code, that is part of the host software stack, has unrestricted access to any resource lent to %the Cloud user.

Cloud computing providers enable even small organizations to deploy web-based services quickly, with low start-up costs, and efficiently adapt the amount of available resources to their current load.
On a machine, the cloud service provider's host divides physical resources among co-located applications from different origins.

While very attractive for their simplicity and adaptability, such services raise important confidentiality and integrity concerns that limit their adoption[CITE FROM HAVEN].
Cloud services implement a unidirectional isolation model where privileged code is protected from the unprivileged one, and retains unrestricted access to the application's data.
In other words, Cloud users are required to entrust the entire privileged software stack with their sensitive data.
Seeing as privacy and confidentiality are growing concerns, new techniques need to be devised to protect the application's data from a potentially compromised or malicious host.\\

Haven[CITE] is a system that leverages Intel software guard extension (SGX) to provide shielded execution of unmodified legacy applications on a commodity OS (windows).
Shielded execution allows to protect the \emph{confidentiality} and \emph{integrity} of a program's data from the host (e.g., the OS) on which it runs.
Confidentiality ensures that the program's intermediate states are not observable by the rest of the system, while integrity guarantees that if the program completes, the output is correct.

Haven runs unmodified applications that do not incorporate in-depth defense against the host, i.e., they were written assuming that the underlying system is well-behaved.
In order to preserve their integrity and protect against Iago attacks, i.e., a malicious host modifying the results of system calls to subvert an application, Haven integrates a library OS (LibOS) based on Microsoft Drawbridge[CITE] that exposes the Windows 8 API.
The application runs on top of the LibOS, which in turns interacts with the untrusted host.

\adrien{Should say SGX allows confidentiality and integrity, integrity also guaranteed by libOS.}

\subsection{The SGX extension}
What is the SGX, what's an enclave, what can you do with it.
Special pages that cannot be accessed or that trigger notification if accessed from outside of the enclave.
How much pages can you put in there?\\

Conclusion of this part must be that it's useful but cannot be enough to provide shielded execution.

\subsection{Design}
The entire "stack" of Haven.
This complicated thing where they have the special untrusted runtime as well.
Things that must be explained:
\begin{itemize}
	\item picoprocess (to protect the host from the guest)
	\item the enclave
	\item the library operating system.
	\item the shield module (how does it verify that host is actually correctly serving a system call?)
	\item untrusted runtime
	\item SGX driver and drawbridge host.
\end{itemize}

\subsection{Evaluation}
Important to see the overhead introduced.
There is a security insentive, but the performance is also important because it is sacrificed here.
Running unmodified binaries is really an advantage.
Disk is still a limitation.
\subsection{Discussion}
%Untrusted time
%VM exits.
%Cannot move as easily as a VM.
%Performance.
%Monitoring is hard when needed.

Still untrusted parts like the time.
Vm exits are a problem.
Cannot move it as easily as a VM, which is a problem since they target cloud deployement.
The performance penalty is pretty high, it improves security but degrades application performance significantely.
Cannot monitor, when on cloud services this is actually required and very important.\\

What we can take away from this paper is actually that applications deployment changed significantely.
Before, we could deploy on our own trusted servers.
Now multiple mutually untrusted applications are colocated on the host and we have this new bidirectional lack of trust between host and guest.
We need to address this issue.
Relying on hardware is an interesting solution because, compared to software, it is less amenable to bugs and hard to attack or circumvent.
The whole libOS thing is also interesting because it shows that by reimplementing the abstractions that regular applications rely on, we can run unmodified applications.