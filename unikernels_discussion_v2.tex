Mirage eschews backward compatibility to enable aggressive specialization of Cloud appliances.
Reimplementing the libraries as well as the application in a single language allows the optimization of the entire user stack at once and achieve a high level of specialization.
In the context of Cloud deployments, we can wonder how desirable backward compatibility really is.
VMs or containers deployed in the Cloud are often single-purpose appliances, providing a well-defined service.
Reimplementing such services in a higher-level language seems feasible, especially if the language already provides external libraries with similar functionalities, e.g., database or network stacks.
This would, however, still require slight modifications of parts of the libraries to use the interface exposed by the underlying hypervisor.

By targeting a virtualization platform, instead of a custom kernel like the exokernel, unikernels sacrifice flexibility for portability.
On one hand, Xen allows Mirage appliances to target a large set of platforms and to co-locate standard VMs and unikernels.
On the other hand, Mirage appliances interact with a high level abstraction provided by the hypervisor that restricts their freedom in terms of physical resource management.
The impact of this restriction on the achievable performance is, however, hard to evaluate.

With the next paper, we study how a system based on the same fundamental design principles as the exokernel and the unikernel leverages modern hardware features to provide bi-directional isolation between a host and a guest.
%With the next paper, we study how modern hardware, coupled with a careful system design, enables to implement %bi-directional isolation between a host and a guest.%
%We also highlight the design principles shared between this solution and the exokernel and unikernel %architectures.


