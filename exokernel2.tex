\section{Exokernel: An Operating System Architecture for Application-Level Resource Management}

In traditional monolithic operating systems, applications interact with the kernel via interfaces that hide machine resources behind fixed high-level abstractions such as processes, files, virtual memory, and interprocess communications (IPC).
Unfortunately, as these abstraction cannot be replaced by untrusted user code, their implementations need to be general enough to serve heterogeneous applications.
Consequently, applications that do not need a certain features still incur its associated overhead.
More generally, the lack of flexibility imposed by fixed kernel abstractions impedes applications performance, flexibility, and functionality, by preventing domain-specific optimizations of resource management, the modification of existing abstractions, and requiring new ones to be inefficiently emulated on top of a rigid high-level kernel interface.

Guided by the \emph{end-to-end} principle[CITE], the authors of the Exokernel paper present an ingenious operating system architecture that provides application-level management of physical resources.
In this new model, the kernel securely exports all hardware resources to untrusted, application-specific library operating system.
Library operating systems are then free to efficiently implement their own system objects and management policies.
In the following sections, we focus on the design principles that are at the heart of the Exokernel architecture and report the evaluation of Aegis, a prototype Exokernel.

\subsection{Design Principles}
The main challenge in designing an Exokernel is to enforce a clean separation between protection and mechanisms[CITE].
The kernel avoids resource management up to the extent required by protection.
It is responsible for securely multiplexing hardware resources and allowing different library operating systems (and applications) to coexist on the same machine.
It must do so with little overhead and without imposing design restrictions that might hurt the application's performance.

The role of the Exokernel is therefore to:
\begin{enumerate*}
	\item \label{expose} expose all hardware resources and track ownership throughout the system,
	\item \label{protect} protect resource usage, and
	\item \label{revoke} revoke access to resources
\end{enumerate*}.
Guided by these design principles, the authors made the design choice to rely on physical names to implement three different mechanisms: secure bindings, visible revocation, and an abort protocol.

The use of physical names in allocation and revocation is a central design choice that has several benefits.
First, physical names allow the kernel interface to be as close as possible to the hardware.
Second, it avoids adding an additional level of indirection, e.g., virtual names, that would introduce overheads and force the kernel to perform a complex management of resources.
Third, it preserves the application's ability to take full advantage of the underlying hardware, e.g., by requesting particular physical resources.

To securely expose resources and track ownership, the Exokernel relies on \emph{secure bindings}.
This protection mechanism binds a specific resource to an application at \emph{bind time}, and installs the entries needed to perform efficient ownership checks at \emph{access time}.
Due to the heterogeneity of the multiplexed resources, the Exokernel relies on different techniques to implement secure bindings:
\begin{enumerate*}
	\item hardware support
	\item software caching, and
	\item downloading application code in the kernel
\end{enumerate*}.

Ideally, specific hardware features should be leveraged whenever possible to implement protection and access time checks.
For example, if a hardware memory TLB is available, the kernel first validates a virtual-to-physical memory mapping, e.g., by performing authorization checks on the physical page, and then installs an entry inside the hardware TLB (bind time), which subsequently performs efficient access time checks.

When hardware support is not possible, or to improve the overall system performance, secure bindings can be implemented as a software TLB, i.e., the kernel can cache bindings.
Following the previous example, the kernel can cache extra virtual to physical mappings inside a large software TLB to speed-up the validation and installation of a new hardware TLB entry.

Finally, downloading code in the kernel allows to perform access checks closer to the application's logic, even when it is not scheduled.
The Exokernel authors rely on this technique to multiplex the network and implement packet filtering.

Visible resource revocation involves the application in the process of reclaiming resources.
While it incurs more latency than the invisible revocation approach taken by traditional operating systems, visible revocation presents two main advantages.
First, it does not deny the application's right to efficient resource management.
Second, it does not require the Exokernel to understand how resources are used and more specifically what parts of the current state should be saved.
Resource revocation is an essential mechanism in any operating system, especially in the presence of scarce resources.
Having an efficient and safe protocol for it is therefore vital.

Faced with an uncooperative library operating system, the Exokernel must be able to break secure bindings and reclaim resources.
After an unsuccessful visible revocation phase, the kernel notifies the library operating system that a hard deadline to comply was set.
If the application still fails to satisfy this requirement, the Exokernel reclaims the resource and informs the application that the secure binding was removed.
Another option would be to kill any non-compliant application.
This alternative, however, would hurt application developers that "\textit{have great difficulty reasoning about hard real-time bounds}"(SIC).
Phrased differently, this abort protocol allows the application to recover from a temporary failure to comply with the Exokernel requirements.
Additionally, the Exokernel provides the application with \emph{repossession vectors} to specify beforehand which resources must be saved during an abort protocol.
The Exokernel also guarantees for each application a set of resources that will not be repossessed during an abort protocol and that can be used by vital parts of the program, e.g., exception handlers.
Finally, an emergency protocol allows to swap out an application, i.e., to repossess all resources without "killing" the application.

\subsection{Aegis Performance}

%Aegis
%CPU
%Exception
%Protected control transfer.

%ExOS
%IPC
%virtual memory
%remote communication. 
In this section we report and review the original evaluation of Aegis, a prototype exokernel, and ExOS, a library operating system.
Since the paper was published in 1995 and Aegis evaluated on the hardware available back then, we ignore the measured numerical values and focus on the exokernel performance compared to Ultrix, a traditional monolithic kernel.

Aegis multiplexes the CPU by dividing it into time slices allocated to the different applications.
Timer interrupts are delivered to the application, which in turn performs the context-switch.
To achieve fairness, applications that do not behave well are forced to give up future slices.
This scheduling mechanism is both flexible and performant.
The cost of a standard procedure call is similar to the one achieved by Ultrix.
For system calls, the authors report that Ultrix's \lstinline{getpid} (one of the fastest syscalls) is an order of magnitude slower than Aegis' slowest one.

In accord with the exokernel's design principles, Aegis dispatches all exceptions to applications.
After handling an exception, the application is not required to re-enter the kernel before resuming its execution.
As a result, Aegis's exception mechanism is more than five times faster than the then state-of-the-art fastest implementation \cite{DBLP:conf/asplos/ThekkathL94}.

Aegis provides primitives for synchronous and asynchronous protected control transfer, a mechanism that can be relied upon to implement efficient IPC abstractions.
Compared with the then fastest published implementation \cite{DBLP:conf/sosp/Liedtke93}, Aegis mechanism is 6.6 times faster.
This is mostly due to the succinct (30 instructions) implementation of the mechanism in the exokernel.

ExOS is a prototype library operating system developed to run on top of Aegis.
ExOS implements an IPC abstraction on top of Aegis' protected control transfer that is more than an order of magnitude faster than Ultrix equivalent implementation.
A simple pipe implementation is 10 times faster than in Ultrix, while a LRPC is 40 to 60 times faster.
This discrepancy is due to the fact that Ultrix needs to emulate new abstractions, such as LRPC, on top of existing ones, e.g., pipes or signals.

The authors then proceed with a comparison between the unoptimized implementation of virtual memory management in ExOS to the one provided in Ultrix.
First, a simple benchmarks allows to show that a standard implementation that does not leverage the exokernel features and Ultrix perform indistinguishably.
A set of microbenchmarks shows that ExOS performs reasonably, i.e., 1.1 to 1.6 times slower, compared to Ultrix on tests that change access rights to contiguous address ranges.
At the same time, ExOS outperforms Ultrix on page-protection traps and more heterogeneous benchmarks by being at least an order of magnitude faster.

\subsection{Discussion}
Good because same design choices than Haven.
In one case it's for security, in the other for performance.
So actually possible to reconcile the two of them.
The main issue now, considering cloud, is how to actually migrate these things?
Also extra advantage can be taken from the fact that the application is one stack (with the libOS).
Authors focused on the fact that your app can be built against a set of default libraries.
But we can do a lot more than that. Now that it forms a whole, we can actually apply PL techniques to it.
That's gonna be the next part.
\adrien{
Cite Sel4 somewhere in the proposal or next section
Finish with the fact that now since we have one stack we can use PL technique.
Since the exokernel is small, we can verify it as well with other techniques e.g., Sel4
How to deploy? Does not solve Haven previous problem}



