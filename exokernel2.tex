\section{Exokernel: An Operating System Architecture for Application-Level Resource Management}

In traditional monolithic operating systems, applications interact with the kernel via interfaces that hide machine resources behind fixed high-level abstractions such as processes, files, virtual memory, interprocess communications (IPC), and file systems.
Unfortunately, as these abstraction cannot be replaced by untrusted user code, their implementations need to be general enough to serve heterogeneous applications.
Consequently, applications that do not need a certain features still incur their associated overhead costs.
More generally, the lack of flexibility imposed by fixed kernel abstractions impedes applications performance, flexibility, and functionality, by preventing domain-specific optimizations of resource management, the modification of existing abstractions, and requiring new ones to be inefficiently emulated on top of a rigid high-level kernel interface.

Guided by the \emph{end-to-end} principle[CITE], the authors of the Exokernel paper present an ingenious operating system architecture that provides application-level management of physical resources.
In this new model, the kernel securely exports all hardware resources to untrusted, application-specific library operating system.
Library operating systems are then free to efficiently implement their own system objects and management policies.
In the following sections, we focus on the design principles that are at the heart of the Exokernel architecture and report the evaluation of Aegis, a prototype Exokernel.

\subsection{Design Principles}
The main challenge in designing an Exokernel is to enforce a clean separation between protection and mechanisms[CITE].
The kernel avoids resource management up to the extent required by protection.
The kernel is responsible for securely multiplexing hardware resources and allowing different library operating systems (and applications) to coexist on the same machine.
It must do so with little overhead and without imposing design restrictions that might hurt the application's performance.

The role of the Exokernel is therefore to:
\begin{enumerate*}
	\item \label{expose} expose all hardware resources and track ownership throughout the system,
	\item \label{protect} protect resource usage, and
	\item \label{revoke} revoke access to resources
\end{enumerate*}.
Guided by these design principles, the authors made the design choice to rely on physical names to implement three different mechanisms: secure bindings, visible revocation, and an abort protocol.

The use of physical names in allocation and revocation is a central design choice that has several benefits.
First, physical names allow the kernel interface to be as close as possible to the hardware.
Second, it avoids adding an additional level of indirection, e.g., virtual names, that would introduce overheads and force the kernel to perform a complex management of resources.
Third, it preserves the application's ability to take full advantage of the underlying hardware, e.g., by requesting particular physical resources.

To securely expose resources and track ownership, the Exokernel relies on \emph{secure bindings}.
This protection mechanism binds a specific resource to an application at \emph{bind time}, and installs the entries needed to perform efficient ownership checks at \emph{access time}.
Due to the heterogeneity of the multiplexed resources, the Exokernel relies on different techniques to implement secure bindings:
\begin{enumerate*}
	\item hardware mechanisms
	\item software caching, and
	\item downloading application code in the kernel
\end{enumerate*}.

Ideally, specific hardware features should be leveraged whenever possible to implement protection and access time checks.
For example, if a hardware memory TLB is available, the kernel first validates a virtual to physical memory mapping, e.g., by performing authorization checks on the physical page, and then installs an entry inside the hardware TLB (bind time), which subsequently performs efficient access time checks.

When hardware support is not possible, or to improve the overall system performance, secure bindings can be implemented as a software TLB, i.e., the kernel can cache bindings.
Following the previous example, the kernel can cache extra virtual to physical mappings inside a large software TLB to speed-up the validation and installation of a new hardware TLB entry.

Finally, downloading code in the kernel allows to perform access checks closer to the application's logic, even when it is not scheduled.
The Exokernel authors rely on this technique to multiplex the network and implement packet filtering. 











\subsection{Aegis Performance}

\subsection{Discussion}



