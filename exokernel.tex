\section{Exokernel: An Operating System Architecture for Application-Level Resource Management}

\subsection{Overview}
Traditional monolithic operating systems run as a single process within one address space.
On x86 processors, the ring model allows to differentiate kernel mode (ring 0) and user mode (ring 3).
Kernel services execute in kernel mode within an address space that is unaccessible in user mode.
Applications interact with the kernel via an interface that hides machine resources behind fixed high-level abstractions.
Such abstractions include processes, files, virtual memory, and interprocess communication (IPC).
Unfortunately, as these abstractions cannot be replaced by untrusted user code, their implementations needs to be general-enough to serve heterogeneous applications.
Consequently, applications that do not need a certain feature still incur the associated overhead costs.
More generally, the lack of flexibility imposed by the kernel abstractions impedes applications performance, flexibility, and functionality.
It prevents domain-specific optimizations, disallows changes to existing abstractions, and requires new abstractions to be inefficiently emulated on top of the provided ones.

The exokernel operating system architecture[CITE] proposes to solve these limitation by delegating resource management to the application.
In this design, the kernel is responsible solely for securely exporting all hardware resources to user-level operating system libraries, which in turns implement their own higher-level system abstractions.

In the next section, we present the general principles behind the exokernel architecture proposal.
We then proceed with the description of its design, and the proposed prototype implementations, i.e., Aegis and ExOS.
%monolithic
%Problem is that they fix the abstraction and implementation
%general purpose means incurring cost that is not needed sometimes.
%So need a new design.
%
%Presents the exokernel, application level resource management.
%Host only responsible for securely exposing physical resources.
%Protection and mechanisms and end-to-end principle.
%
% Protection/mecha enables to attain the proper level of abstraction.
% Suprisingly going back to lower gives more flexibility and hence allows for higher abstractions and tools to be used within the kernel,
% e.g., oo languages, overloading etc. to easily specialize and extend.
% Kernel doesn't need to understand resources, just needs to check that it is binded properly.
% Protect without understanding.
% 
%
%\adrien{
%	First motivation in Overview.
%
%	Then Principles to solve issues.
%
%	Then general design.
%
%	Then design of Aegis and ExOS.
%
%}
\subsection{Design Principles}
%State the principles.
%Secure bindings.
%Multiplexing everything
%Downloading code.
%Revocation
%abort
The need for application-level management of resources is justified, according to the exokernel authors, by the \emph{end-to-end} argument.
The application knows better than the operating system what resource management decisions impact its own performance.
As a result, it should be given complete freedom over the management of its resources.

At the same time, the kernel still needs to ensure that different applications can coexist on the same machine.
This requires to multiplex physical resources among them in a secured manner.

These requirements are achieved by separating protection from management.
The exokernel focuses on
\begin{enumerate*}
	\item tracking ownership of resources,
	\item protecting resources, and
	\item revoking access to resources.
\end{enumerate*}.
By providing a low-level interface, the exokernel allows applications to implement their own abstractions and resource management as they see fit.
More importantly, by protecting resource usage, it allows different library operating systems, i.e., implementation of higher-level abstractions on top of the exokernel interface, to safely coexist on the same machine.

In order to avoid resource management at all cost, an exokernel needs to securely expose all hardware resources as directly as possible.
Management should only be limited to the strict minimum required by protection.
Library operating systems must be allowed to request specific physical resources, using their physical names directly.
Introducing a level of indirection by using virtual names would force the exokernel to perform resource management and would impede the performance of the overall system and the application.
Finally, revoking resources must be made visible to the library operating system in order to facilitate application level resource management.
This requirement is crucial in the context of sparse resources.

\subsection{The Exokernel Design and Prototypes}
% Exokernel responsible for exposing resources.
% At the heart of everything are secure bindings.
% three techniques: hardware, software caching, and downloading code in kernel.
% checked only at bind time
% make as little expensive as possible.
% rely on hardware.


\subsection{Evaluation}
\subsection{Discussion}

%\subsection{Application knows best}
%end to end
%separation between policies and mechanisms
%
%
%
%Lib OS
%LibOS abstractions can be simpler and more specialized, especially since doesn't need to multiplex resources among competing applications with different demands.
%Can trust the application as well (since only one to be touched by it).
%Less kernel crossing
%Easy to get new standards
%portability if implemented on top of posix or other standards.
%OO and stuff like that to augment the libOS possibilities.
% support for dynamic linking and shared libs would reduce space required.
%
% Exokernel Design
% separate protection from management.
% Goal is to multiplex the physical resources and isolate apps from each other
% 
% 3 tasks: 1) track ownership, 2) ensure protection 3) revoke access to resources.
% secure bindings, visible revocation, abort protocol.
% 
% Design principles
% securely expose hardware: physical memory, CPU, disk memory, TLB, addressing context identifiers.
% interrupts, exceptions, cross-domain calls.
% Well => exokernel should avoid resource management.
% 
% Expose allocation: request specific physical resources.
% Expose Names: expose physical names, and also hardware things like disk position arm, TLB entries etc.
% Expose revocation: good to have physical names for that
% 
% Secure bindings
% quickly executed.
% checks only at bind time => decouples protection from management.
% 3 mechanisms for secure bindings: hardware mecha, software caching, downloading application code.
% Downloading code means kernel event can give control to the code without kernel crossing.
% Downloading code = executing when application is not scheduled: e.g., packet filter (instead of scheduling each and every application).
% Asynchronous events as well (such as message reception etc.)
% 
% Visible resource revocation.
% visible vs. invisible trade-off.
% 
% Abort protocol
% force the LibOS to relinquish some resource.
% repossession vector.
% need to provide some guarantees to LibOS some resources that are the last to be reclaimed.

%Application can manage resources.
%Kernel attack surface is smaller and should be less prone to bugs because it simply safely exposes resources.
%Kernel and application responsabilities are better defined.
%Crossing from user to kernel can be reduced to a strict minimum.
%Avoid all the kernel bypass etc.
%Reusability of libraries?\\
%
%Allows for new abstractions closer to application's logic.
%How hard would it be to implement lwc in exokernel? Not hard at all.\\
%
%relevant to dataplanes as well.\\
%
%How is scheduling implemented?

%\subsection{Evaluation}
%\adrien{Move some of this into previous part}
% Evaluation
% Aegis
% Compares with Ultrix 4.2 (monolithic kernel)
% 
% Processor time slice sharing.
% Requires a little bit of collaboration and I'm not convinced about how efficient and flexible this can be.
%
%Exception context, Interrupt context, Protected entry context, Addressing context(address space identifier, status register, tag => required to switch env.)
%Ultrix exception handling two orders of magnitude greater than Aegis.
%
% Protected control transfer
% yield to another process.
% either synchronous or asynchronous.
% First give current + all future to callee, and asynchro gives only remainder of current.
% 6.6 times faster than state-of-the-art mechanism L3.
% 
% Dynamic packet filter -> don't care.
% 
% ExOS
% focus on IPC, virtual memory, and remote communication
% For Pipes, ExOS can create its own abstractions on top of Aegis, and is more efficient.
% Note: Pipe in user space so can inline calls to read and write and make it more efficient.
% Virtual memory:
% Still pretty good even if not optimized.
% 
% Application safe handlers:
% Downloading code (avoids scheduling app and kernel crossings)
% decouple latency critical tasks from process scheduling.
% Sandboxed 
% 

%Must prove that it improves performance.
%Security aspect of it?

%%\subsection{Discussion}
%Main critic is how can you trust libOS?
%Drivers reimplementation?
%Application code is not enough sufficient, need the libOS as well.
%Can we have a POSIX and Linux default set of LibOSes?
%Debugging requires to understand the LibOS as well, which is bad...
%\\
%
%We can inject haven's solution for example with the set of guaranteed pages.
%The application can put them into an enclave.
%Find an attack scenario from corrupted host, and how we can prevent it.\\
%
%Also goes along the fact that now we use different abstractions.
%Instead of building them on top of POSIX or rigid kernel ones, can implement directly and efficiently, and hence get good performance.
