\section{Unikernels: Library Operating Systems for the Cloud}

\subsection{Overview}
%TO SAY
%Keywords: Specialization (of the kernel itself)
%1. Goal is to have secured, small sized, efficient applications easy to deploy in cloud.
%2. The application is the kernel. Strip down and seal the app to only what is needed.
%3. Relies on PL and compiler techniques, e.g., static analysis, static type-safety etc.
%4. Single address space !!! have to put that somewhere.
This should be a high level overview of the paper.

\textit{Unikernels} are defined as small sized sealed single-purpose appliances that can easily be deployed over cloud services.
More specifically, a unikernel encapsulate an application's logic as well as the required system libraries and language runtime.
However, unlike regular general-purpose VMs, unused functionalities are stripped-away at compile-time therefore yielding a small sized bootable image while reducing the potential attack surface exposed by said image.\\

Describe the overall architecture and design.\\

\subsection{Specialization through a unified common language.}
The application is the OS.
This goes further than before.\\

Specialising benefits are smaller deployable things.
A common lower level interface that relies on the supervisor.
Smaller attack surface.
Reconfiguration require re-compilation so hard to modify behaviour?\\

Single high level language means that we can leverage modern techniques, e.g. type safety, static analysis, compiler optimizations to boost performance.
At the same time, single address space is enough because we have the functionalities from the language that act as barriers.

\subsection{Evaluation}
TODO: redac this\\

Main goals and claims: since supposed to be deployed on the cloud, both size and boot time are relevant.
This is evaluated qualitatively by comparing LOCs between unikernels and VMs, and by measuring boot time required when deploying the appliances.
Result is that can boot in less than 50 ms (how does it compare to Linxu?)\\

Threading is efficient because no kernel and user space separation, hence avoid the mode switch overheads.\\

Network and Storage type safety does not introduce significant delays.
Thanks to single address space, 0-copy for receive so faster.
Transmission is, however, slower because requires extra cpu processing (high level language).\\

Application is DNS.

\subsection{Discussion}

Comparison with exokernel: different approach regarding the abstraction level.
Still trusts the hypervisor, could we use the SGX to protect it somehow?
Argument that hypervisor enables to get compatibility with drivers is good, but still need to reimplement all the system libraries...
Still suffers from the same problems as any LibOS, you need to reimplement the libraries.
Other problem is that you limit the flexibility by requiring them to be implemented in OCaml.
Compatibility with other applications is ensured by having a network communication between the appliance and a VM that runs the service.
Not good because shared memory etc. is not really straight forward.\\

What is better than exokernel? 
The fact that we can rely on high-level pl tools like verification, dead-code elimination etc.
Packageable (which is not really the case for an app in exokernel and not a considered concern).\\

What is not as good?
Everything needs to be implemented in OCaml, because we have the runtime environment and all of that.
In exokernel, you have an interface, and it could be enough to somehow communicate with it.\\

What is bad as well?
Can you actually snapshot an appliance and migrate it?
Check in the paper if this is the case because I'm really not sure.
