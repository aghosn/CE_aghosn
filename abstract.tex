\begin{abstract}
Modern software applications need to process an ever increasing amount of data, satisfy strict service level objectives, while addressing present-day concerns regarding privacy and security.
Application requirements and architectures have become so complex and heterogeneous that no general-purpose monolithic operating system can fully answer their needs, both in terms of performance and security.
Increasingly often, developers free themselves from the legacy rigid abstractions proposed by the operating system to achieve competitive results.

We believe that the respective roles of the kernel and the application should be re-evaluated to address contemporary performance and security challenges.
We further argue that those challenges can both be addressed with the same fundamental design principles that tend to establish a clear separation between the kernel and the application.
Such a design would allow application-specific management of physical resources, a greater design flexibility,  and to improve overall performances. 
This design would moreover simplify the implementation of new security models that address Cloud deployment confidentiality and integrity concerns, e.g., a mutual distrust between a host and a guest.

We study three papers addressing these challenges and highlight their common design choices:
the exokernel \cite{DBLP:conf/sosp/EnglerKO95} enables application-level management of physical resources, unikernels \cite{DBLP:conf/asplos/MadhavapeddyMRSSGSHC13} leverage the same design to create specialized stand-alone secured and high performance appliances for the Cloud, while Haven \cite{DBLP:journals/tocs/BaumannPH15} provides shielded execution of legacy applications.

Finally, understanding the arguments and solutions presented in these papers, we propose our research goal that pertains to kernel design for modern software applications.
\end{abstract}