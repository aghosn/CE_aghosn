% Old but good, solves kernel bypass thing.
% Provides small set of primitives and simple implementation, means efficient and easy to validate (smaller attack surface, less LOC).
% At the same time, we see that the same fundamental principles that were used in Haven are used here. Meaning that mutual distrust could be implemented.
% Due to its performance, we can expect to incur SGX but not other overheads that are present in Haven.
% At the same time, can leverage the fact that application is close to the library os. We need to see what this unveils.
% Finally, still unsolved migration and cloud management/deployement. 
Although originally published two decades ago, the Exokernel addresses modern concerns in terms of system design and application performance.
It allows application-level management of resources by exposing an interface as close as possible to the underlying hardware, hence removing the need for kernel bypass in order to increase performance \cite{BelayPKGKB14,DBLP:journals/tocs/CaoFKL96}.
The Exokernel provides a small set of primitives to securely expose and multiplex hardware resources.
This reduced set of responsibilities allows a compact and simple implementation of the Exokernel that is less amenable to implementation bugs, exposes a smaller attack surface, and is highly efficient.

The Exokernel design revolves around the basic principle of separation between protection and mechanisms to achieve good performance.
The same principle was used in Haven to allow the implementation of bi-directional isolation between a host and a guest.
As a result, we suspect that confidentiality and integrity, as provided in Haven, are not incompatible with achieving good performance.

The Exokernel exposes an interface close to what a Cloud client actually rents from the service provider, i.e., raw hardware resources.
Such a design is therefore also relevant with regards to modern application deployments. 

The Exokernel paper focuses on the design of the kernel itself.
We now want to study how leveraging this design enables the aggressive specialization of the application stack.
More precisely, with the next paper, we study how one can leverage the fact that an application can be linked to a specific libOS in order to improve the overall performance, without sacrificing security.
We further study the benefits of such solutions, coupled with virtualization and modern programming language techniques, for Cloud deployments.