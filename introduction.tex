\section{Introduction}

%1. Applications evolved
%	1.1. Complicated multi-sourced piece of code.
%	1.2. Requirement for performance and security.
%	1.3. Deployed in new ways (e.g., cloud services)
%	
%2. Operating systems are too rigid.
%	2.1. Abstractions have not evolved much in the past decades.
%	2.2. Resource protection as well as management.
%	2.3. Leave little space for application specific logic.
%	2.4. Built in a way that requires them to be trusted.
%	
%3. Objective
%	3.1. Goals is to study designs that allow
%		3.1.1. separate protection from management.
%		3.1.2. Allow application specific management of resources.
%		3.1.3. Symmetric isolation mechanisms (e.g., protect guest from host).
%		3.1.4. Limit attack surface and rely on verification tools.
%	3.2. For that we study 3 different solutions
%		3.2.1 Hardware mechanism with SGX.
%		3.2.2. Kernel design.
%		3.2.3 Software kernel packaged for an application. 

%TODO remove afterwards
%TODO should define that in the paper we say guest for app or vm or anything that runs on top of the host, i.e., the kernel
\subsection{Applications evolved}
Past decades have seen the appearance of new applications.
Available online, spread everywhere, frameworks, libraries.
Multiple source give code that makes current applications ALSO dynamically download code in browser.
This means new security challenges, with fine-grained control over what's happening.
At the same time, explosion of data that circulates, requirement for performance.
Finally, deployed in new ways. Not only running on a personal hardware, but deployed on cloud services, co-located with other applications.
Still requirement for security, isolation, non-trust of the host, performance.\\


%In the past decades, software applications have multiplied, became more complex, and intertwined in new ways.%
%A vast majority of modern applications rely on public API's, external libraries, or software frameworks, hence incorporating code from various, someti%mes untrusted, sources.%
%The web has become the main distribution platform for such software, hence making it impractically hard to verify sources. %TODO reformulate this one.%
%\adrien{Rephrase all of that}The emergence and explosion of such complex systems leads to new security challenges, as verification tools do not allow to ensure a bug free software, applications still need to be executed without being %trusted.
%At the same time, 
%\adrien{Not sure if should keep it or just forget about it}Targeting a framework is much more valuable for attackers as it can potentially affect a large number of applications.
%
%Unavoidable e.g., browsers require to execute javascript, run untrusted web applets etc.
%Browsers need to rely on sandboxing mechanisms to isolate javascript code ran on each page, and download and execute untrusted web applets.
%Another mechanism implemented by certain operating systems consists in maintaining a list of trusted software providers and warn users whenever code from an untrusted source is being executed (Microsoft mitigation mechanism).
%
%Not only did the way applications were developed and distributed changed, but also the way to deploy them.
%Cloud services, for example, host and co-located applications that originate from different sources.
%This new way to deploy applications requires to isolated untrusted applications and prevent them from harming both the underlying host and other co-located applications.
%Still requires to trust the host.
%
%\adrien{Performance???}

\adrien{Need to reformulate everything. Put emphasis on the mutual distrust (don't care about multiple sources) between host and application. Then talk about performance}\\

Cloud computing providers enable even small organizations to deploy web-based services quickly, with low start-up costs, and efficiently adapt the amount of available resources to their current load.
On a machine, the cloud service provider's host divides physical resources among co-located applications from different origins.
While very attractive for their simplicity and adaptability, such services raise important security issues.
From the service provider's point of view, co-locating mutually distrusted applications requires to
\begin{enumerate*}
	\item isolate applications from each other and
	\item prevent an application from corrupting the host
\end{enumerate*}.
As a result, sandboxing mechanisms became a fundamental building block of today's cloud services.
However, most of these solutions assume that the host itself is part of the trust chain.
Most cloud service providers require application developers to trust the hypervisor running on the machine.
\adrien{Take things from Haven}.

\adrien{At the same time, it is not exactly the same as having access to your own hardware.
The performance is tunable to the extent of the provided service abstractions.}

\textit{Summary}: Applications more and more complex, deployed in new ways, require performance.

\subsection{Operating systems did not}
Meanwhile, operating system abstractions did not evolve.
Unix was designed in the seventeens, and is a source of inspiration for the current operating systems.
At the same time, it's the main source of inspiration for POSIX.
Problem is, these abstraction do not take into account new challenges.
That's why some papers break POSIX compatibility, or simply by-pass the kernel.
NEED EXAMPLES (IX can be one for dataplanes, LWC for more flexible memory management).\\

Kernel is responsible for resource protection and management at the same time.
While protection makes sense, i.e., role is to share the resources and ensure each unit some amount,
management seems to be bad, because the kernel is not aware of what the application needs.
A good example is garbage collection (okay I need to find a paper about this, maybe in the exokernel).\\

Finally kernels are all powerful. A compromised kernel can hurt any application.
While denial of service seems unavoidable, what might be a problem is data leakage, corruption, or highjacking of an application.
With more an more colocated applications on cloud services, this is a real problem.
Past efforts were made to protect host from guest, but now we actually want some guarantees the other way around as well.\\

\textit{Summary}: kernel abstractions did not follow the same evolution as applications.
There's a requirement for more liberty in resource management for the application, we want to separate protection from management.
At the same time, we also need to protect guest from host.

\subsection{Objectives}
 In this paper, we want to study existing designs that could better answer today's requirements imposed by the way applications are developped and deployed.
 We want the kernel to focus on protection rather than management.
 The application's knows best how to manage its resources, and should therefore be given access to such resources to do with as they want.
 By leaving the application with most of the management, we de-involve the kernel and hence get a better separation between kern and application, that allows to more easily provide a symmetric isolation/protection mechanisms from host to guest and guest to host.
 We also want to study how modern software tools and techniques can impact kernel design.
 How can we leverage software verification, static analysis and all of that in kernels? 
 C and C++ are the most used languages, we might want a completely type safe language.\\

 \textit{Summary:}This requires modifictions at several levels.
 We study a hardware solution, a kernel one, and a software one.
 We first present the following papers, and then the research proposal.


 
