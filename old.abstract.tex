\begin{abstract}

Modern applications incorporate code from various, sometimes unverified, sources.
As a result, both trusted and untrusted code cohabit within an application, henceforth requiring fine-grained isolation and protection mechanisms at boundaries known only to the application's developer.
At the same time, modern applications strive for performance, which often entails to manage resources in a way that follows the application's logic.

\adrien{High level funcitonalities and too involved in the app execution.}
Unfortunately, mainstream operating systems expose rigid general purpose abstractions that often limit performance, flexibility, and functionalities.
Moreover, by failing to separate resource protection from management, kernels are responsible for functionalities vital to the application, such as access to the network or virtual memory management, thus making them an overpowered single point of failure.
At a time when more and more applications are deployed over cloud services, entrusting the kernel, and more generally any privileged software, with such critical functionalities is not only inefficient, but also a risk.

In this paper, we study solutions that re-define the respective responsibilities of the kernel and the application, giving back its freedom to the latter. 
We rely on three publications: Haven, the Exokernel, and the Unikernel paper.
Haven leverages the Intel SGX extension to shield a guest system from an untrusted host.
The exokernel operating system allows application-level management of physical resources.
Unikernels are compile-time specialized standalone kernels for single purpose appliances deployed over cloud platforms.

Finally, understanding the arguments and solutions presented in these papers, I propose my research topic that relates to resource management and the improvement of kernel's role and abstractions to better address modern challenges.
\end{abstract}