\documentclass[10pt,journal,a4paper]{IEEEtran}


\usepackage{amsmath}                        % extended math env
\usepackage{amssymb}                        % more math symbols
% \usepackage{balance}                      % balance coloumns on final page
\usepackage{booktabs}                       % nicer tables + hline thickness
% \usepackage{caption}                        % control of caption sizing
\usepackage[usenames,dvipsnames]{color}     % color access

\usepackage{comment}
\let\labelindent\relax
\usepackage[inline]{enumitem}

\usepackage{listings}
\usepackage[inline]{enumitem}
\usepackage[bottom]{footmisc}

\newcommand{\adrien}[1]{\noindent{\color{Green} {\bf \fbox{AG} {\it#1}}}}
% *** CITATION PACKAGES ***
%
\ifCLASSOPTIONcompsoc
  % IEEE Computer Society needs nocompress option
  % requires cite.sty v4.0 or later (November 2003)
  % \usepackage[nocompress]{cite}
\else
  % normal IEEE
  % \usepackage{cite}
\fi


% *** GRAPHICS RELATED PACKAGES ***
%
\ifCLASSINFOpdf
  % \usepackage[pdftex]{graphicx}
  % declare the path(s) where your graphic files are
  % \graphicspath{{../pdf/}{../jpeg/}}
  % and their extensions so you won't have to specify these with
  % every instance of \includegraphics
  % \DeclareGraphicsExtensions{.pdf,.jpeg,.png}
\else
  % or other class option (dvipsone, dvipdf, if not using dvips). graphicx
  % will default to the driver specified in the system graphics.cfg if no
  % driver is specified.
  % \usepackage[dvips]{graphicx}
  % declare the path(s) where your graphic files are
  % \graphicspath{{../eps/}}
  % and their extensions so you won't have to specify these with
  % every instance of \includegraphics
  % \DeclareGraphicsExtensions{.eps}
\fi

\providecommand{\PSforPDF}[1]{#1}


% NOTE: PDF hyperlink and bookmark features are not required in IEEE
%       papers and their use requires extra complexity and work.
% *** IF USING HYPERREF BE SURE AND CHANGE THE EXAMPLE PDF ***
% *** TITLE/SUBJECT/AUTHOR/KEYWORDS INFO BELOW!!           ***
\newcommand\MYhyperrefoptions{bookmarks=true,bookmarksnumbered=true,
pdfpagemode={UseOutlines},plainpages=false,pdfpagelabels=true,
colorlinks=true,linkcolor={black},citecolor={black},pagecolor={black},
urlcolor={black},
pdftitle={How to write an EDIC thesis proposal},%<!CHANGE!
pdfsubject={Typesetting},%<!CHANGE!
pdfauthor={EDIC},%<!CHANGE!
pdfkeywords={EDIC, candidacy exam, EPFL}}%<^!CHANGE!



% correct bad hyphenation here
\hyphenation{op-tical net-works semi-conduc-tor}


\begin{document}

\title{Reviewing kernels \& applications roles}



% The paper headers
\markboth{EDIC Research Proposal}%
{Shell \MakeLowercase{\textit{et al.}}: EDIC Research Proposal}


\IEEEcompsoctitleabstractindextext{%

\begin{abstract}
Modern software applications need to process an ever increasing amount of data, satisfy strict service level objectives, while addressing present-day concerns regarding privacy and security.
Application requirements and architectures have become so complex and heterogeneous that no traditional general-purpose operating system can fully answer their needs, both in terms of performance and security.
Increasingly often, developers free themselves from the legacy rigid abstractions proposed by the operating system to achieve competitive results.

We believe that the respective roles of the kernel and the application should be re-evaluated to address contemporary performance and security challenges.
We further argue that those challenges can both be addressed with the same fundamental design principles that tend to establish a clear separation between the kernel and the application.
Such a design would allow application-specific management of physical resources, a greater design flexibility,  and to improve overall performances. 
This design would moreover simplify the implementation of new security models that address Cloud deployment confidentiality and integrity concerns, e.g., a mutual distrust between a host and a guest.

We study three papers addressing these challenges and highlight their common design choices:
the exokernel \cite{DBLP:conf/sosp/EnglerKO95} enables application-level management of physical resources, unikernels \cite{DBLP:conf/asplos/MadhavapeddyMRSSGSHC13} leverage the same design to create specialized stand-alone secured and high performance appliances for the Cloud, while Haven \cite{DBLP:journals/tocs/BaumannPH15} provides shielded execution of legacy applications.

Finally, understanding the arguments and solutions presented in these papers, we propose our research goal that pertains to kernel design for modern software applications.
\end{abstract}

\begin{IEEEkeywords}
thesis proposal, candidacy exam write-up, EDIC, EPFL
\end{IEEEkeywords}}

% make the title area
\maketitle

\IEEEdisplaynotcompsoctitleabstractindextext

\IEEEpeerreviewmaketitle



%\section{Introduction}
\section{Introduction}

%1. Applications evolved
%	1.1. Complicated multi-sourced piece of code.
%	1.2. Requirement for performance and security.
%	1.3. Deployed in new ways (e.g., cloud services)
%	
%2. Operating systems are too rigid.
%	2.1. Abstractions have not evolved much in the past decades.
%	2.2. Resource protection as well as management.
%	2.3. Leave little space for application specific logic.
%	2.4. Built in a way that requires them to be trusted.
%	
%3. Objective
%	3.1. Goals is to study designs that allow
%		3.1.1. separate protection from management.
%		3.1.2. Allow application specific management of resources.
%		3.1.3. Symmetric isolation mechanisms (e.g., protect guest from host).
%		3.1.4. Limit attack surface and rely on verification tools.
%	3.2. For that we study 3 different solutions
%		3.2.1 Hardware mechanism with SGX.
%		3.2.2. Kernel design.
%		3.2.3 Software kernel packaged for an application. 

%TODO remove afterwards
%TODO should define that in the paper we say guest for app or vm or anything that runs on top of the host, i.e., the kernel
\subsection{Applications evolved}

Applications and traditional operating system designs and abstractions evolved at different paces over the past decades, leading to an important mismatch that impedes both security and performance of modern software.
In an increasingly multi-tasked and web-oriented environment, application developers strive for ever higher network throughputs, finer-grained memory management, fast access to heterogeneous external devices, and strong isolation guarantees, all of which seems hardly achievable with the common standards (e.g., POSIX) defined decades ago.
This mismatch lead to ad-hoc re-implementations of system services, closer to the application's logic and able to achieve either more flexibility or higher performances.
Such solutions include user-level reimplementation of threads[CITE], IPC mechanisms[CITE], memory abstractions[CITE], or the networking stack[CITE].
As a result, modern applications are built on top of multiple layers of software frameworks and libraries, rather than on top of the available OS abstractions.

%At the same time, the complexity inherent to the inclusion of code from various sources, as well as new deployment practices (e.g., Cloud services) and increasing concerns for privacy calls for a re-evaluation of operating systems' security models.
At the same time, aging operating system designs fail to address contemporary security concerns.
Present-day software incorporate code from various, potentially untrusted, sources.
Consequently, sandboxing mechanisms became a fundamental building block of application's development.
For example, web-browsers rely on such techniques to prevent web-applets, browser plug-ins, and web-pages's JavaScript from harming the host.
Unfortunately, the lack of flexible system abstractions lead to heterogeneous and complex implementations of sandboxes.


Meanwhile, the widespread hierarchical security model that tends to protect a privileged host from an unprivileged guest is unable to fully answer present privacy concerns.
Nowadays, applications generate and process sensitive user data that should only be available to the application.
Regrettably, privileged code has unrestricted access to unprivileged code's resources, based on the assumption that it is part of the trusted software stack.
This assumption, however, fails to address new deployment paradigms (e.g., Cloud services), where little is known about the software running on the host.
Even worse, growing concerns have been voiced against user data aggregation by mainstream operating systems[CITE microsoft].
Mutual distrust, between host (privileged code) and guest (unprivileged code) hence seems to appear better suited to address modern security and privacy requirements. 

%%The advent of the Internet area lead to new application deployment paradigms and an incredible increase in the amount of sensitive user data transiting throughout the web.
%%Cloud service providers enable even small companies to quickly deploy web-applications at a low starting cost.
%%Applications from multiple, mutually distrusted sources, are co-located on the same machine.


\subsection{Operating systems did not}
\adrien{Use the paper sent by Marios for everything that is related to POSIX abstractions and new layered application.
shift of paradigm where applications mostly interface with frameworks that either interact with os or by-pass it to re-implement some low-level features such as network.}

Meanwhile, operating system abstractions did not evolve.
Unix was designed in the seventeens, and is a source of inspiration for the current operating systems.
At the same time, it's the main source of inspiration for POSIX.
Problem is, these abstraction do not take into account new challenges.
That's why some papers break POSIX compatibility, or simply by-pass the kernel.
NEED EXAMPLES (IX can be one for dataplanes, LWC for more flexible memory management).\\

Kernel is responsible for resource protection and management at the same time.
While protection makes sense, i.e., role is to share the resources and ensure each unit some amount,
management seems to be bad, because the kernel is not aware of what the application needs.
A good example is garbage collection (okay I need to find a paper about this, maybe in the exokernel).\\

Finally kernels are all powerful. A compromised kernel can hurt any application.
While denial of service seems unavoidable, what might be a problem is data leakage, corruption, or highjacking of an application.
With more an more co-located applications on cloud services, this is a real problem.
Past efforts were made to protect host from guest, but now we actually want some guarantees the other way around as well.\\

\textit{Summary}: kernel abstractions did not follow the same evolution as applications.
There's a requirement for more liberty in resource management for the application, we want to separate protection from management.
At the same time, we also need to protect guest from host.

\subsection{Objectives}
 In this paper, we want to study existing designs that could better answer today's requirements imposed by the way applications are developped and deployed.
 We want the kernel to focus on protection rather than management.
 The application's knows best how to manage its resources, and should therefore be given access to such resources to do with as they want.
 By leaving the application with most of the management, we de-involve the kernel and hence get a better separation between kern and application, that allows to more easily provide a symmetric isolation/protection mechanisms from host to guest and guest to host.
 We also want to study how modern software tools and techniques can impact kernel design.
 How can we leverage software verification, static analysis and all of that in kernels? 
 C and C++ are the most used languages, we might want a completely type safe language.\\

 \textit{Summary:}This requires modifictions at several levels.
 We study a hardware solution, a kernel one, and a software one.
 We first present the following papers, and then the research proposal.


 


\section{Shielding Applications from an Untrusted Cloud with Haven}

Cloud computing service providers enable even small organizations to deploy web-based services quickly, with low start-up costs, and efficiently adapt the amount of available resources to their current load.
On a machine, the cloud service provider's host divides physical resources among co-located applications from different origins.

While attractive for their simplicity and adaptability, such services raise important confidentiality and integrity concerns that limit their adoption[CITE FROM HAVEN].
Cloud services implement a unidirectional isolation model where privileged code (the host) is protected from the unprivileged one (the guest), and retains unrestricted access to the application's data.
In other words, Cloud users are forced to entrust the entire Cloud privileged management software stack with their sensitive data and intellectual property.
Seeing as privacy and confidentiality are growing concerns, new techniques need to be devised to protect the application's data from a potentially compromised or malicious host.

Haven[CITE] is a system that leverages Intel Software Guard Extension (SGX) to provide shielded execution of unmodified legacy applications on a commodity OS (windows).
Shielded execution allows to protect the \emph{confidentiality} and \emph{integrity} of a program from the host (e.g., the OS) on which it runs.
Confidentiality ensures that the program's intermediate states are not observable by the rest of the system, while integrity guarantees that if the program completes, the output is correct.
We briefly introduce the Intel SGX extension main features before presenting Haven's design.

\subsection{The SGX extension}

%General
%Protection of access.
%CPU keeps metadata about the pages (put EPC and SECS here and CPU enclave mode)
% SGX attestation.
% 
% Only trusted part is the CPU itself. 
The Intel Software Guard Extension (SGX) defines a set of new CPU instructions that allows to create and manipulate \emph{enclaves}.
An enclave is a portion of the user address space protected against unauthorized access or modification by software, regardless of its privilege level.
SGX ensures data confidentiality, integrity, and prevents replay attacks on the enclave's content.
In this scheme, the processor acts as the root of trust to achieve the protected execution of enclaved programs.

Enclave's code and data become available once the processor \emph{enters} the enclave.
Entering an enclave requires the processor to switch to \emph{enclave mode}, load the appropriate page mappings, and jump to a specific entry point contained within the enclave.
While cache resident data is protected by the table lookaside buffer (TLB), SGX relies on cryptographic techniques to ensure the confidentiality and integrity of an enclave's page in main memory.

Main memory, i.e., DRAM, is not part of the trusted computing base (TCB).
As a result, in enclave mode, all traffic between the CPU and main memory needs to be encrypted and integrity protected.
SGX relies on a dedicated hardware unit, called \emph{Memory Encryption Engine} (MEE), that is part of the TCB and operates as an extension of the Memory Controller (MC).
The MEE is in charge of a portion of the physical memory, i.e., the \emph{protected region}, determined at boot time, that holds enclave pages, the processor's metadata used to handle them, and the region's integrity tree.
The MEE implements and enforces the encryption and integrity protection schemes for enclave resident data.
More specifically, it decrypts and verifies the integrity of enclaves data loaded from main memory, and encrypts and computes MAC tags and version attributes to protect the integrity of (and prevent replay attacks on) data written to main memory.
A failure to verify an enclave's data integrity results in a fault that blocks the processor's MC and requires a machine reboot.

The processor stores enclave pages and their metadata in the \emph{Enclave Page Cache} (EPC), a subrange of the protected region.
Each enclave is associated with a corresponding \emph{SGX Enclave Control Structure} (SECS) that resides within the EPC and is used by the hardware to validate interactions with enclave pages.
The SECS tracks physical pages that belong to a particular enclave, their types, their corresponding virtual mappings, as well as their access rights.
The EPC also acts as a cache holding free enclave pages and allows to dynamically extend the memory allocated to a specific enclave.

SGX implements local and remote attestation mechanisms to verify the integrity of a newly created enclave.
Enclaves are instantiated by untrusted software.
As a result, an extra verification step is required to prove that both software and hardware environments were initialized as expected.
Local attestation enables one \emph{reporting} enclave to prove its genuineness to a \emph{target} enclave via a symmetric key cryptographic scheme.
The reporting enclave triggers a procedure taking as argument the target's enclave identity.
The procedure generates a \emph{report} containing a MAC of itself signed by the processor with the target's secret key.
A report encapsulates both hardware and software environments, referred to as the \emph{enclave's identity}.
The report is then sent to the target enclave that can verify its authenticity by recomputing the signed hash and comparing it with the one provided.

Remote attestation relies on asymmetric cryptography to produce a report.
First, the local attestation is used with a special \emph{quoting enclave}.
The quoting enclave then generates and inserts a MAC in the report computed with a private, device specific key and sends the report to the remote entity.
The known public key can then be used to verify the authenticity of the MAC contained within the report.

\subsection{Design}

Haven leverages the low level isolation mechanisms provided by SGX to achieve shielded execution of legacy application binaries.
Since the host OS cannot be trusted, Haven provides most of the operating system services required by the application and manages policies and mechanisms for virtual resources, while the host retains management over physical ones.
This clear separation, coupled with defensive programming and careful interactions between guest and host, is supposed to facilitate Haven's resistance to Iago attacks, i.e., a malicious host exploiting the application's trust to achieve its goals.

Haven's definition of shielded execution requires to
\begin{enumerate*}
	\item bootstrap and validate the execution environment within the enclave,
	\item implement virtual resource management within the enclave,
	\item defend against a misbehaving host by verifying the result of every service requested from it, and
	\item prevent the host from gathering information on the enclave's state via exception handlers
\end{enumerate*}.
We now describe how Haven's design addresses each of these concerns.
Figure [REF] provides an overview of Haven's runtime stack.
The application, the LibOS, and the shield module form the trusted software base.
They are all encapsulated inside the enclave.

Haven wraps the application in a sandbox provided by Drawbridge[CITE].
The application executes within a secured container, called \emph{picoprocess}, that exposes a narrow ABI to request OS services.
As a standard sandboxing mechanism, the picoprocess protects the host from the guest.

The application relies on a trusted library operating system (LibOS) to request system services.
In Haven, the LibOS is a modified version of Windows 8, retrofitted into a set of libraries, that implements the full OS API.
The LibOS is under user control and can be inspected and verified by the application.

In order to protect the application from Iago attacks, Haven relies on a \emph{shield module}.
The shield module exposes the drawbridge ABI to the LibOS and implements kernel functionalities and primitives such as memory management and thread synchronization.
It isolates the LibOS from a malicious host by validating all parameters and results that cross the border between trusted and untrusted parts of the runtime.
Such validations depend on the specific service required from the host OS, but mostly consist in sanity checks and observably consistent results.

The untrusted interface at the boundary of the enclave was designed to allow correctness verification by the shield, while being flexible enough to permit an efficient implementation of services on both sides.
In order to achieve both security and performance, Haven's authors decided to decouple policies from mechanisms.
The host is responsible for managing physical resources while the guest handles virtual ones.

Underneath the untrusted interface, an untrusted runtime module exists to forward calls between the guest and the host.
This module also acts as a bootstrap to create the enclave and load the shield.

The host OS interacts with the enclave via the Drawbridge ABI and provides services relating to memory management, e.g., changing memory access rights or allocating untrusted virtual memory, thread signaling, access to external untrusted devices, e.g., storage, and to the system time.

Applications are deployed on the Cloud as encrypted disk images that contain both the application and LibOS binaries.
The Cloud service provider is responsible for creating the picoprocess and loading the untrusted runtime, which in turns creates the enclave and loads the shield module.
SGX remote attestation mechanism is used to compute a quote of the shield and verify its integrity as well as the enclave's environment.
Afterwards, the shield generates asymmetric cryptography keys, securely sends its public key to a machine controlled by the user along with the quote.
The user verifies the shield's integrity, encrypts the disk image with the public key, and sends it to the shield.
The application is then ready to be loaded.

Exception handling requires the host's intervention and must therefore be done with care to avoid leaking information about the enclave's state.
When an exception occurs in enclave mode, SGX saves the content of the registers and information about the fault within the current TCS.
A synthetic context for the exception is then created to prevent data leaks to the host.
This synthetic context is passed to the host OS which can either handle the exception, or report it back to the guest.
In the latter case, the shield verifies the legitimacy of the exception before reporting it to the LibOS.

Haven stacks several software layers, each existing to provide specific functionalities to the ones above it.
This layered architecture, coupled with the defensive programming design adopted, might introduce non-negligible overheads.
The next section focuses on Haven's reported performance.

%The  requirements for Haven imply that the application must be able to verify and trust its entire software stack that is part of the enclave, that services still provided by the underlying host must be treated as potentially misbehaving,  
%Drawbridge running on top of host kernel is responsible for the thing. 
%%Insert image.
%
% Rely on drawbridge to sandbox the application and have a trusted LibOS (trusted by the app itself).
% Exposes a narrower interface, and is verifiable by application. 
% Reducing interface means less ways to exploit and easier to control.
% 
% Shield module within the TCB. Gives core OS operations
% Also a bootloader for libos and application.
% There to protect against Iago attacks.
% 
% Untrusted interface (not in enclave)
% Policy mechanism separation
% Guest is responsible for virtual, host for physical
% Untrusted runtime
% create enclave, loading shield, forwarding calls to host
% Decided to prevent app from allocating anything outside of the enclave.
% Storage?
% 
% Threads and synchronization
% user level scheduling
% 
% Deployement is weird because also need to verify the VHD.
% Basically load and attest shield, then send the disk image.
% 
% Switching in and out of enclave requires specific mechanisms (clearing registers etc.)
% Requires copy for parameters passed by reference
% 
% Exception handler not really interesting.
% 
% 
% Requires to load all binaries directly because no demand loading possible.
\subsection{Evaluation}

The authors were provided with a functional SGX emulator to implement Haven.
Unfortunately, such an emulator cannot be relied on to measure, or even approximate, the system performance.
As a result, a second version of Haven that introduces synthetic delays for SGX instructions was used for instead.
The performance evaluation hence consists in two steps:
\begin{enumerate*}
	\item compare the performance of Haven with other deployments, where SGX instructions have a null cost and enclave crossings only incur a TLB flush, and
	\item vary the synthetic delays for SGX main features, i.e., enclave crossings, dynamic memory allocation, and accesses to the EPC
\end{enumerate*}.

The first step of the evaluation allows to measure overheads that are purely due to Haven's implementation.
The second step models the performance impact of the SGX hardware.
By cross-referencing Haven's reported results with a recent performance evaluation of SGX\cite{DBLP:conf/IEEEwisa/ZhaoSTZX16}, we are able to get a better understanding of Haven's actual performance.

The first evaluation relies on two real-world applications, namely, a TPC-E benchmark running on a Microsoft SQL server, and an Apache HTTP server benchmark.
Haven's performance is compared with the throughputs achieved by running the same experiments natively, in a hyper-V VM, and in Drawbridge.
As a general result, Haven's throughput is inferior to all other configuration for both applications.
On the database benchmark, it incurs a 13\% slowdown compared to Drawbridge.
On the HTTP server, the same comparison yields a 40\% slowdown.
We note however that Haven running with a dedicated filesystem outperforms Drawbridge, due to the read-intensive nature of the workload that access cached pages inside the enclave.

The second step of the evaluation introduces synthetic delays for enclave crossings, dynamic memory allocation, and accesses to the EPC.
Varying the access latency for EPC pages is done by changing the DRAM clock rate.
Slowing down the memory by a third introduced a 21\% slowdown for the database application, and a 7\% one for the HTTP server.
According to the SGX performance evaluation in \cite{DBLP:conf/IEEEwisa/ZhaoSTZX16}, memory operations inside the enclave are more than three times slower than outside of it.
It therefore seems that Haven's authors largely underestimated the cost of accessing the EPC.

The authors note that while enclave crossings and dynamic allocation of memory both have impacts on the HTTP server application's throughput, the database application seems to be sensitive only to enclave crossings.
By plugging-in the SGX performance evaluation result from \cite{DBLP:conf/IEEEwisa/ZhaoSTZX16}, i.e., 7 kcycles per crossing, we see that enclave crossings induce a 20\% extra slowdown in throughput in the previously measured Haven's database experiment.
For the HTTP server, the actual slowdown is of the order of 1\% for enclave crossings.

More generally, Haven's authors predicted a 31\% to 54\% slowdown in performance compared to a virtual machine solution.
As mentioned above, memory operations on EPC pages were largely underestimated.
A thorough experiment, on real hardware, is needed to better understand how Haven really compares to VM based solutions.
We also note that, while targeting Cloud services, Haven does not provide the same flexibility as VMs in terms of deployment and migration of applications.


\subsection{Discussion}

%Tries to answer present day problems with an approach that relies on hardware support.
%Seems legit.
%Design principles at the heart of haven or policy mechanism separation, bi-directional mistrust,
%involving the host as little as possible.
%
%But performance issues,
%not a complete system since still requires to trust the host for system time for example.
%Question is, adopting similar design at the heart of the security aspect, can we do better.
Haven addresses the modern confidentiality and integrity concern that sprung from the growth of Cloud services.
In the context of rented resources, users would like to protect the confidentiality and integrity of their data when executing their applications on machines controlled by an untrusted entity.
Haven leverages hardware mechanisms provided by Intel SGX and well-defined design principles to provide shielded execution of legacy binaries.
Haven enforces policies and management separation to disentangle the guest from the host, obtain a cleaner separation between the mutually distrustful domains, and more easily monitor their interactions.
At a high-level, the host is reduced to exposing physical resources, while the guest manages virtual ones.

Unfortunately, as we have seen in the evaluation section, Haven sacrifices performance for confidentiality and integrity.
Parts of the overheads are directly imputable to the hardware mechanisms implemented by Intel SGX and cannot be avoided.
In particular, we observed that memory operations on the EPC are non-negligibly slow.
At the same, Haven's stack itself significantly decreases the application's performance compared to alternate deployments.

The open question is to determine if performance and confidentiality/integrity can be reconciled under a common set of design principles.
In other words, we wonder if the design choices, at the heart of Haven's design, that enforce the isolation between host and guest, preclude performance.
The next paper suggests that it is quite the opposite.






\section{Exokernel: An Operating System Architecture for Application-Level Resource Management}

In traditional monolithic operating systems, applications interact with the kernel via interfaces that hide machine resources behind fixed high-level abstractions such as processes, files, virtual memory, and interprocess communications (IPC).
Unfortunately, as these abstraction cannot be replaced by untrusted user code, their implementations need to be general enough to serve heterogeneous applications.
Consequently, applications that do not need a certain features still incur its associated overhead.
More generally, the lack of flexibility imposed by fixed kernel abstractions impedes applications performance, flexibility, and functionality, by preventing domain-specific optimizations of resource management, the modification of existing abstractions, and requiring new ones to be inefficiently emulated on top of a rigid high-level kernel interface.

Guided by the \emph{end-to-end} principle[CITE], the authors of the Exokernel paper present an ingenious operating system architecture that provides application-level management of physical resources.
In this new model, the kernel securely exports all hardware resources to untrusted, application-specific library operating system.
Library operating systems are then free to efficiently implement their own system objects and management policies.
In the following sections, we focus on the design principles that are at the heart of the Exokernel architecture and report the evaluation of Aegis, a prototype Exokernel.

\subsection{Design Principles}
The main challenge in designing an Exokernel is to enforce a clean separation between protection and mechanisms[CITE].
The kernel avoids resource management up to the extent required by protection.
It is responsible for securely multiplexing hardware resources and allowing different library operating systems (and applications) to coexist on the same machine.
It must do so with little overhead and without imposing design restrictions that might hurt the application's performance.

The role of the Exokernel is therefore to:
\begin{enumerate*}
	\item \label{expose} expose all hardware resources and track ownership throughout the system,
	\item \label{protect} protect resource usage, and
	\item \label{revoke} revoke access to resources
\end{enumerate*}.
Guided by these design principles, the authors made the design choice to rely on physical names to implement three different mechanisms: secure bindings, visible revocation, and an abort protocol.

The use of physical names in allocation and revocation is a central design choice that has several benefits.
First, physical names allow the kernel interface to be as close as possible to the hardware.
Second, it avoids adding an additional level of indirection, e.g., virtual names, that would introduce overheads and force the kernel to perform a complex management of resources.
Third, it preserves the application's ability to take full advantage of the underlying hardware, e.g., by requesting particular physical resources.

To securely expose resources and track ownership, the Exokernel relies on \emph{secure bindings}.
This protection mechanism binds a specific resource to an application at \emph{bind time}, and installs the entries needed to perform efficient ownership checks at \emph{access time}.
Due to the heterogeneity of the multiplexed resources, the Exokernel relies on different techniques to implement secure bindings:
\begin{enumerate*}
	\item hardware support
	\item software caching, and
	\item downloading application code in the kernel
\end{enumerate*}.

Ideally, specific hardware features should be leveraged whenever possible to implement protection and access time checks.
For example, if a hardware memory TLB is available, the kernel first validates a virtual-to-physical memory mapping, e.g., by performing authorization checks on the physical page, and then installs an entry inside the hardware TLB (bind time), which subsequently performs efficient access time checks.

When hardware support is not possible, or to improve the overall system performance, secure bindings can be implemented as a software TLB, i.e., the kernel can cache bindings.
Following the previous example, the kernel can cache extra virtual to physical mappings inside a large software TLB to speed-up the validation and installation of a new hardware TLB entry.

Finally, downloading code in the kernel allows to perform access checks closer to the application's logic, even when it is not scheduled.
The Exokernel authors rely on this technique to multiplex the network and implement packet filtering.

Visible resource revocation involves the application in the process of reclaiming resources.
While it incurs more latency than the invisible revocation approach taken by traditional operating systems, visible revocation presents two main advantages.
First, it does not deny the application's right to efficient resource management.
Second, it does not require the Exokernel to understand how resources are used and more specifically what parts of the current state should be saved.
Resource revocation is an essential mechanism in any operating system, especially in the presence of scarce resources.
Having an efficient and safe protocol for it is therefore vital.

Faced with an uncooperative library operating system, the Exokernel must be able to break secure bindings and reclaim resources.
After an unsuccessful visible revocation phase, the kernel notifies the library operating system that a hard deadline to comply was set.
If the application still fails to satisfy this requirement, the Exokernel reclaims the resource and informs the application that the secure binding was removed.
Another option would be to kill any non-compliant application.
This alternative, however, would hurt application developers that "\textit{have great difficulty reasoning about hard real-time bounds}"(SIC).
Phrased differently, this abort protocol allows the application to recover from a temporary failure to comply with the Exokernel requirements.
Additionally, the Exokernel provides the application with \emph{repossession vectors} to specify beforehand which resources must be saved during an abort protocol.
The Exokernel also guarantees for each application a set of resources that will not be repossessed during an abort protocol and that can be used by vital parts of the program, e.g., exception handlers.
Finally, an emergency protocol allows to swap out an application, i.e., to repossess all resources without "killing" the application.

\subsection{Aegis Performance}

%Aegis
%CPU
%Exception
%Protected control transfer.

%ExOS
%IPC
%virtual memory
%remote communication. 
In this section we report and review the original evaluation of Aegis, a prototype exokernel, and ExOS, a library operating system.
Since the paper was published in 1995 and Aegis evaluated on the hardware available back then, we ignore the measured numerical values and focus on the exokernel performance compared to Ultrix, a traditional monolithic kernel.

Aegis multiplexes the CPU by dividing it into time slices allocated to the different applications.
Timer interrupts are delivered to the application, which in turn performs the context-switch.
To achieve fairness, applications that do not behave well are forced to give up future slices.
This scheduling mechanism is both flexible and performant.
The cost of a standard procedure call is similar to the one achieved by Ultrix.
For system calls, the authors report that Ultrix's \lstinline{getpid} (one of the fastest syscalls) is an order of magnitude slower than Aegis' slowest one.

In accord with the exokernel's design principles, Aegis dispatches all exceptions to applications.
After handling an exception, the application is not required to re-enter the kernel before resuming its execution.
As a result, Aegis's exception mechanism is more than five times faster than the then state-of-the-art fastest implementation \cite{DBLP:conf/asplos/ThekkathL94}.

Aegis provides primitives for synchronous and asynchronous protected control transfer, a mechanism that can be relied upon to implement efficient IPC abstractions.
Compared with the then fastest published implementation \cite{DBLP:conf/sosp/Liedtke93}, Aegis mechanism is 6.6 times faster.
This is mostly due to the succinct (30 instructions) implementation of the mechanism in the exokernel.

ExOS is a prototype library operating system developed to run on top of Aegis.
ExOS implements an IPC abstraction on top of Aegis' protected control transfer that is more than an order of magnitude faster than Ultrix equivalent implementation.
A simple pipe implementation is 10 times faster than in Ultrix, while a LRPC is 40 to 60 times faster.
This discrepancy is due to the fact that Ultrix needs to emulate new abstractions, such as LRPC, on top of existing ones, e.g., pipes or signals.

The authors then proceed with a comparison between the unoptimized implementation of virtual memory management in ExOS to the one provided in Ultrix.
First, a simple benchmarks allows to show that a standard implementation that does not leverage the exokernel features and Ultrix perform indistinguishably.
A set of microbenchmarks shows that ExOS performs reasonably, i.e., 1.1 to 1.6 times slower, compared to Ultrix on tests that change access rights to contiguous address ranges.
At the same time, ExOS outperforms Ultrix on page-protection traps and more heterogeneous benchmarks by being at least an order of magnitude faster.

\subsection{Discussion}

% Old but good, solves kernel bypass thing.
% Provides small set of primitives and simple implementation, means efficient and easy to validate (smaller attack surface, less LOC).
% At the same time, we see that the same fundamental principles that were used in Haven are used here. Meaning that mutual distrust could be implemented.
% Due to its performance, we can expect to incur SGX but not other overheads that are present in Haven.
% At the same time, can leverage the fact that application is close to the library os. We need to see what this unveils.
% Finally, still unsolved migration and cloud management/deployement. 
Although originally published two decades ago, the Exokernel addresses modern concerns in terms of system design and application performance.
It allows application-level management of resources by exposing an interface as close as possible to the underlying hardware, hence removing the need for kernel bypass in order to increase performance \cite{BelayPKGKB14,DBLP:journals/tocs/CaoFKL96}.
The Exokernel provides a small set of primitives to securely expose and multiplex hardware resources.
This reduced set of responsibilities allows a compact and simple implementation of the Exokernel that is less amenable to implementation bugs, exposes a smaller attack surface, and is highly efficient.

The Exokernel design revolves around the basic principle of separation between protection and mechanisms to achieve good performance.
The same principle was used in Haven to allow the implementation of bi-directional isolation between a host and a guest.
As a result, we suspect that confidentiality and integrity, as provided in Haven, are not incompatible with achieving good performance.

The Exokernel exposes an interface close to what a cloud client actually rents from the service provider, i.e., raw hardware resources.
Such a design is therefore also relevant with regards to modern application deployments. 

The Exokernel paper focuses on the design of the kernel itself.
We now want to study how leveraging this design enables the aggressive specialization of the application stack.
More precisely, with the next paper, we study how one can leverage the fact that an application can be linked to a specific libOS in order to improve the overall performance, without sacrificing security.
We further study the benefits of such solutions, coupled with virtualization and modern programming language techniques, for cloud deployments.




\section{Unikernels: Library Operating Systems for the Cloud}
Traditionally, applications are deployed in the Cloud as virtual machines (VM).
A VM packs a complete operating system, such as Linux or Windows, that allows to run unmodified application processes.
Typically, a VM deployed in the Cloud can be viewed as a single-purpose appliance, i.e., it consists of a guest operating system running a main application, e.g., a database or web server, that relies on smaller services running in parallel.
Due to the lack of standards for application configuration, VMs often run custom scripts as glue code to initialize the execution environment.

Although destined to execute a single main application, Cloud VMs are insufficiently specialized.
They contain an entire operating system image, distributed with some default services.
Apart from unnecessarily increasing the disk image size for this single-purpose appliance, hence requiring more resources to deploy and run the VM, the presence of unwanted default services increases the potential attack surface exposed to a malicious entity and threatens the integrity of the VM.

The Unikernel paper[CITE] presents a new model to deploy applications in the Cloud that addresses the above-mentioned issues by radically specializing the Cloud application's image.
\emph{Unikernels} are small sized sealed single-purpose appliances entirely written in a high-level programming language, that can easily be configured and deployed over Cloud platforms.


\subsection{Design Principles}
The Unikernel revisists the idea of library operating systems (libOS) by applying it to Cloud platforms.
A unikernel is a standalone kernel that encapsulates a single-purpose application destined to be deployed over the Cloud.
As such, unikernels execute on top of a VM hypervisor, e.g., Xen[CITE].
The advantages of targeting a virtualization platform instead of a custom kernel are twofold.
First, the hypervisor provides a fixed virtual hardware abstraction that alleviates the need to take into account heterogeneous hardware devices.
The hypervisor provides hardware drivers for a large number of devices and abstracts them away behind a fixed high-level interface.
Second, the hypervisor virtual abstraction facilitates both vertical and horizontal dynamic scaling.

Unikernels achieve aggressive specialization by relying on a single high-level programming language.
System libraries and the application are implemented in a common, statically typed, programming language, and compiled as a single specialized unikernel.
The compiler performs static analysis on the entire user stack and optimizes the whole system at once.
This enables to:
\begin{enumerate*}
	\item perform most of the appliance static configuration at compile time,
	\item reduce the final image's size, and
	\item improve the application's security
\end{enumerate*}, all of which without sacrificing performance.

Static configuration of a unikernel appliance happens at compile time.
As mentioned previously, standard VMs incur a non-negligible deployment overhead.
Custom scripts are executed to configure and start all services needed by the application, e.g., a database for a web server, and glue together independent packages or applications.
Unikernels, on the other hand, integrate all dependencies as libraries, compiled within the application itself.
Build parameters passed to the compilation configuration enable static configuration, while dynamic configuration is achieved via specific function calls to the library.

Unikernels are highly specialized and compact binaries.
Since all the libraries are linked and compiled against the application itself, static analysis techniques such as dead-code elimination (DCE) can be applied to reduce the final image's size.
The authors claim that in most cases, the final binary's size is on the order of kilobytes, and hence easily deployable across the Internet.
Furthermore, in the context of Cloud platforms, where resources are rented, small size binaries consume less resources to host the program's code and data, and hence minimize expenses.

By eschewing backward compatibility and requiring the entire stack to be written in a single high-level typed programming language, unikernels can leverage type-safety as well as static analysis and verification tools to improve the application's security.
First, DCE reduces the attack surface exposed to a remote attacker system-wide.
Unused libraries as well as unreachable portions of code are not included in the final binary and cannot be leveraged by an attacker to subvert the application.
Second, a unikernel executes entirely within a single address space and relies on the language's type safety to enforce access control and restrictions.
The single-address space model eases the integration of security techniques. 
For example, to prevent code injection, the unikernel is initialized such that no page mapping is both executable and writable.
To further protect against possible attacks, the unikernel can issue a \emph{seal} hypercall, that prevents future page table modifications.
Other techniques and code instrumentation, e.g., address space randomization, stack canaries and guard pages, can also easily be added.

While type safety and system-wide optimizations are attractive, unikernels represent a daunting engineering challenge.
All libraries and protocols used by the application need to be rewritten in the same high-level language.
An alternate solution is to leverage interoperability at the network level.
Services that are not reimplemented in the chosen language can be encapsulated in separate VMs and communicate with the unikernel via standard network protocols.
Unfortunately, this technique increases resource usage and prevents system-wide optimizations.
Finally, we note that parts of the system, such as the garbage collector, are type-unsafe and still require to be protected against. 

A common argument against type-safe and memory managed programming languages is the overhead incurred by type checks and garbage collection.
Performance is critical to Cloud services renters to achieve their service level objectives.
According to the authors, unikernels performances are comparable to standard VMs.
We report the evaluation results presented in [CITE] in the next section.

\subsection{Prototype and Evaluation}
%implementation called Mirage.
%Implemented in OCaml
%Hypervisor is Xen.
%Single process, threading achieved via lwthreads part of the language.
%typed protocol I/O give details from p465.
%Language runtime slightly modify, special support for device drivers based on I/O (not sure if going to talk about it).
%
%Evaluation performs microbenchmarks, DNS Server for safe network stack, Openflow control appliance, web server and database (storing and networking), and respective sizes.
%
%Boot time: less than half time required by linux debian running apache 2 in paravirtualized mode.
%Modifying how Xen builds domains enables to reboot the machine in less than 50 milliseconds.
%
%threads creation: better than pv and native. Creating 20 millions is more than twice faster in mirage os than pv, and noticeably faster than native. heap allocation.
%also more precised thread timers, mostly due to no user/kernel space crossings
%
%Net + Storage
% 10% overhead for mirage due to type safety.
% TCPv4 comparison on par with Linux.
%\adrien{Maybe sum it up better, say that microbenchmarks prove that no performance overhead is introduced.
%Then say general comment about the real life applications, and say we focus on bind because it shows two things: improved perf, and better security due to type-safety.}
%The authors evaluate a prototype evaluation of unikernels, called Mirage, that leverages OCaml system programming and static type system to produce appliances running on top of a Xen[CITE] hypervisor.
%A Mirage application is a standalone kernel, encapsulating a single process running in a 64-bit address space.
%Concurrency within a VM is achieved via the OCaml Lwt cooperative threading[CITE] library.
%At the same time, a \emph{domainpoll} function enables to block the VM for external events and timeouts.
%Lightweight threads enable to implement I/O protocols in a type-safe, non-blocking manner.
%Network processing as well as storage I/O are implemented as unikernel libraries under the control of the application that rely on a unified device driver model to read/write from/to a device.
%Mirage applications contain a guest VM driver implemented in OCaml that communicates with a Xen backend driver via events and a shared memory page that holds requests and responses for device operations.
%This shared memory ring is at the heart of every I/O in Mirage and allows device drivers to be implemented as OCaml libraries linked against the application.
%The Mirage runtime memory management was specialized to dedicate segment to these external I/O pages.
%
%Mirage's performance evaluation consists in microbenchmarks, to evaluate standalone features such as boot time and the network stack, and more realistic applications, namely a DNS server, and Openflow appliance, and a web server %backed by a database.
%We briefly report the results from[CITE].
%
%Mirage appliances exhibit satisfactory performance results in the microbenchmark evaluation.
%The Mirage image time to boot is more than twice smaller than an equivalent Linux kernel running as a paravirtualised Xen domU executing a similar application.
%Regarding thread creation, Mirage manages to create 20 millions of threads more than twice faster than the paravirtualized linux, and slightly faster than linux running natively.
%Surprisingly, the single address space model used by Mirage allows less jitter in thread timers than linux-based configurations, mostly due to the absence of user/kernel crossings.
%Next, the evaluation of the TCPv4 stack implementation in Mirage shows that it performs similarly to the Linux 3.7 TCPv4 one, while providing a type-safe implementation of the protocol.
%Finally, Mirage performs the same as a the paravirtualized linux with direct I/O (no buffer) for a simple random read throughput test.
%
%For more realistic applications, Mirage's performance is at least equivalent (and sometimes superior) to state-of-start implementations of the same functionalities.
%The Mirage DNS Server appliance achieves 75-80 kqueries/s, while NSDN, a high performance implementation, reaches a max throughput of 70 kqueries/s.
%Apart from an increased throughput, Mirage provides a type-safe implementation that is not amenable to most of the past 10 years reported bugs in the Bind software.\\
%\adrien{Re-writing}
The authors evaluate a prototype unikernel implementation, called Mirage, that leverages OCaml system programming and static type system to produce appliances running on top of a Xen[CITE] hypervisor.
A Mirage application is a standalone kernel, encapsulating a single process running in a 64-bit address space.
Concurrency within a VM is achieved via the OCaml Lwt cooperative threading[CITE] library.
At the same time, a \emph{domainpoll} function enables to block the VM for external events and timeouts.
Lightweight threads enable to implement I/O protocols in a type-safe, non-blocking manner.
Network processing as well as storage I/Os are implemented as unikernel libraries under the control of the application and rely on a unified device driver model to read/write from/to a device.
Mirage applications contain a guest VM driver implemented in OCaml that communicates with a Xen backend driver via events and a shared memory page that holds requests and responses for device operations.
This shared memory ring is at the heart of every I/O in Mirage and allows device drivers to be implemented as OCaml libraries linked against the application.
The Mirage runtime memory management was specialized to dedicate a segment to these external I/O pages.

Mirage main features are evaluated against a set of microbenchmarks.
Compared to a Linux kernel running as a paravirtualized Xen domU, a Mirage appliance boots more than two times faster, enables the creation of 20 millions threads twice as fast, incurs less jitter in thread timers thanks to the absence of user/kernel crossings, and performs network and storage I/O without significant overheads.

Mirage is then evaluated on more realistic applications, namely, a DNS server, an OpenFlow controller appliance, and a web server backed by a database.
In the remainder of this section, we focus on the DNS server evaluation that exhibits most advantages of relying on Mirage to deploy Cloud applications.
We note, however, that in all experiments, Mirage performance is close to (or even better than) state-of-the-art implementations.

The Mirage DNS server application includes libraries for the network stack and an in-memory filesystem.
The authors compare the Mirage appliance to a state-of-the-art high performance implementation, NSD 3.2.10.
Adding memorization of responses enabled the Mirage appliance to reach a throughput lying between 75 and 80 kqueries/s.
In comparison, NSD reaches a maximal throughput of 70 kqueries/s.
Apart from the performance evaluation, the authors highlight the benefits of having a type-safe implementation of a DNS server.
Reportedly, the Bind software (a mature and popular DNS server) suffered from 40 important vulnerabilities in the past 10 years, mostly related to memory management errors and poor handling of exceptional cases.
Mirage's type-safety prevents this kind of errors by design.
The DNS server Mirage appliance is a perfect example showing that type-safety, and more generally improved security, do not  preclude performance. 

To conclude, as promised, Mirage yields small size binary images, even for realistic applications.
For example, the Mirage DNS server image is 183.5 kB, while the equivalent Linux image, striped down to the used components, is 462 MB.


\subsection{Discussion}

Haven and Mirage are diametrically opposed: where the former strives to support legacy applications, the latter eschews backward compatibility.
Reimplementing the libraries as well as the application in a single language allows to optimize the entire user stack at once and achieve a high level of specialization.

In the context of Cloud deployments, we can wonder how desirable backward compatibility really is.
VMs or containers deployed on the Cloud are often single-purpose appliances, providing a well-defined service.
Reimplementing such services in a higher-level language seems feasible, especially if the language already provides external libraries with similar functionalities, e.g., database or network stack.
This would, however, still require to slightly modify parts of the libraries to use the interface exposed by the underlying hypervisor.

By targeting a virtualization platform, instead of a custom kernel like the Exokernel, unikernels sacrifice flexibility for portability.
They interact with a high level abstraction provided by the hypervisor that restricts their freedom in terms of physical resource management.
At the same time, Xen allows Mirage appliances to target a large set of platforms and to co-locate standard VMs and unikernels.
It is hard to compare the achievable performance of a unikernel and an Exokernel based solution.
First, modern virtualization is efficient.
Second, the unikernel retains a lot of flexibility in terms of management of virtualized resources.
Finally, unikernel appliances are highly optimized.
\adrien{Sel4 somewhere}


\section{Research Proposal}
%Cloud is there to rent infrastructure. So just expose physical resources.
%What is my research proposal?
%
%We see the same problems appearing again and again, while the LibOS seems to come back everytime.
%Maybe it is time to realize that the current situation is no longer viable.
%That the separation between application and kernel needs to be redefined.
%That, by applying the end-to-end principle, the application knows best how to manage resources.\\
%
%At the same time, we have a growing distrust between host and guest because of the explosion of cloud services.
%Allowing the kernel to have complete power over resource management is just not good anymore.
%Most of the efforts done in the past was to shield host from guest, but now the opposite is also needed.
%See how many compromised cloud services (Find a cool example of such situation).\\
%
%Proposition is to reevalute the design and respective role of kernel and application such that we can both provide bidirectional isolation while allowing applications to achieve better performance by %removing limitations due to system libraries implementations.
%Believe that this could provide good contribution in security, by allowing sandboxes to rely on hardware support to implement their logic, for example, instead of emulating functionalities on-top of the %OS.

%	Monolithic kernels come to an end.
%	Flexibility is achieved by either by-passing it, or working within a virtualized environment.
%	First semester project say a little bit.
%
%	New technologies show that we're at a crossroad, new challenges because the model is not the same.
%	Needs performance, hardware improves but end of moore's law, more and more cooperation between different machines.
%	Fixing communication between them impacts their performance.
%	Exokernel gives a nice design, but see that a hybrid solution like unikernels is more flexible.
%	How redefining kernel interface, how much virtualization can be put there?
%	Smaller kernel means smaller surface, makes it practical to use verification tools on the kernel itself.
%
%	Apply high-level programming techniques. Managed memory for kernel objects?
%	High-jack reference counting? Typed memory systems. Ownership tracking.

Traditional monolithic kernels fail to answer today's challenges, in terms of performance, security, and ease of deployment.
To achieve higher throughput, greater flexibility, or improved security, researchers and developers rely on kernel by-passes[CITE], virtualization, or software frameworks.
The rigid kernel abstractions devised 50 years ago for portability and principled software development need to be revisited[CITE] to take into account the heterogeneity of modern systems and requirements.
Providing a new system abstraction, such as light-weight contexts[CITE], in a monolithic kernel, requires to either modify an existing kernel[CITE], inefficiently emulate it on top of the kernel, or rely on virtualization techniques.
The latter solution was the object of my first semester project.
By leveraging hardware support for virtualization, i.e., Intel VTX[CITE], I managed to implement light weight contexts without modifying the Linux kernel.
I however quickly reached performance and flexibility limitations, and had to rely on several tricks, e.g., loading parts of the program at different addresses, to improve the overall performance of my implementation.

% Crossroads, with new challenges that require to rethink systems from the ground up.
% Moore's Law is coming to an end, power achieved by relying on several machines, need efficient communication.
% Do less but do it well.
% The way we consider resources.
% Deployment changed, resources are rented, 
% At the same even types of computations changed. Need to process large amounts of data, find correlations.
% This already impacts how the hardware is designed, GPUs GPA, and should as well inpact systems.
%
% As there is no  
% This requires efficient management, communication, and cooperation solutions that are hard to implement on top of intrinsically rigid systems.
Moore's law is coming to an end while the volume of data that we need to process keeps increasing.
Modern set ups harvest the cumulative power of dozens of machines to achieve their computations in a timely fashion.
At the same time, the types of computations performed also evolved, both in terms of volumes and complexity.
\adrien{FPGAs}
This impacted the design and usage of modern hardware: GPGPUs are leveraged to speed-up parallel data processing, new types of processors are designed, e.g, graph analytic processors[CITE].
Operating systems need to adapt to these new paradigms quickly, and this requires to study alternate designs.
Due to heterogeneous requirements, no general-purpose monolithic system can pretend to provide satisfiable performance for all types of workloads.
\adrien{merge first paragraph here and split in two}
In the continuation of the Exokernel, systems should therefore focus on securely exposing the available resources.

% My research proposal focuses on the design of modern systems that need to take into account the following challenges:
% 1. Performance requirements and application specific management of resources.
% 2. The increasingly parallel and complex hardware set up.
% 3. a new security model where physical resources might be rented or adversary software might execute on the same.
% 
% Explore the space between a low level interface, and hardware supported virtualization to find a sweet spot abstraction that does not preclude performance.
% Push modern hardware features, such as SGX, and modern programming languages tools to quantify performance and security gains.
% Ownership model into kernel resources, typed memory for kernel objects and bookeeping.
My research proposal focuses on the study of modern systems, designed to answer present day challenges, i.e.,
\begin{enumerate*}
	\item achieve high performance to allow applications to meet service level objectives,
	\item consider the shift in paradigms such as increasingly parallel and complex hardware setups, and
	\item answer present security requirements, such as mutual distrusts between a system and an application running on top of it.
\end{enumerate*}
To do so, we need to explore the design space between a low level interface, as in the Exokernel, and virtualized environments.
The ideal solution would be a hybrid design that strikes the sweet spot between exposing raw hardware resources and providing a stable and portable abstraction.

This design would act as a foundation for further exploration of new solutions on higher-level components. 
More precisely, new abstractions could easily be created to leverage the latest hardware features, such as Intel SGX, and create modern execution environments that answer present days security requirements.

As shown in the Unikernel paper, operating systems can benefit from advanced programming languages features, such as memory safety, type safety, intrinsic support for concurrency, or ownership tracking.
Several papers[CITE] already explored how type-safe memory managed languages can be used in the operating system.
With the rise in popularity of new paradigms and models, exported by languages like Rust and its ownership mechanism, or Go and its communicating-sequential-processes (CSP) style for concurrent programming, the space of possible designs and solutions has yet to be explored.








\section{conclusion}
Applications have evolved, both in the way they are written, and the way they are executed and deployed.
Need new solutions and to redefine the role of operating systems.
Library operating systems proposed a design that can be modified to both enable management of resources closer to the application's logic, as well as including new features that allow mutual distrust between the host and the guest in the case of applications deployed over the cloud.
More than that, sandboxing, dataplanes, user-level implementation of system libraries, all of them can be rendered more efficient by simply providing a common base kernel that exposes flexible abstractions.
We know that the problem is the huge reimplementation of both libs and drivers, as well as extending the amount of untrusted code (i.e., the libraries in the OS themselves).
Taking a step by step approach as done with projects based on Dune is a first thing.
Can the same efforts done by Microsoft can be done for Linux? 
Sandboxes are going to become more and more complicated if the os does not become simpler.
More and more kernel bypass is also to be expected.
Virtualization and kernel modules allows to do just this, but still incure some overhead.

\begin{thebibliography}{1}

\bibitem{mittelbach}
F.~Mittelbach, M.~Goossens, J.~Braams, D.~Carlisle, and C.~Rowley, 
\emph{The {\LaTeX}} Companion, 2nd~ed.\hskip 1em plus
0.5em minus 0.4em\relax Addison-Wesley Professional, 2004.

\end{thebibliography}

% biography section
% 
% If you have an EPS/PDF photo (graphicx package needed) extra braces are
% needed around the contents of the optional argument to biography to prevent
% the LaTeX parser from getting confused when it sees the complicated
% \includegraphics command within an optional argument. (You could create
% your own custom macro containing the \includegraphics command to make things
% simpler here.)
%\begin{biography}[{\includegraphics[width=1in,height=1.25in,clip,keepaspectratio]{mshell}}]{Michael Shell}
% or if you just want to reserve a space for a photo:

%\begin{IEEEbiography}{Michael Shell}
%Biography text here.
%\end{IEEEbiography}

% if you will not have a photo at all:
%\begin{IEEEbiographynophoto}{John Doe}
%Biography text here.
%\end{IEEEbiographynophoto}

% insert where needed to balance the two columns on the last page with
% biographies
%\newpage

%\begin{IEEEbiographynophoto}{Jane Doe}
%Biography text here.
%\end{IEEEbiographynophoto}

% You can push biographies down or up by placing
% a \vfill before or after them. The appropriate
% use of \vfill depends on what kind of text is
% on the last page and whether or not the columns
% are being equalized.

%\vfill

% Can be used to pull up biographies so that the bottom of the last one
% is flush with the other column.
%\enlargethispage{-5in}



% that's all folks
\end{document}


