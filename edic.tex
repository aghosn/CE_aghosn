\documentclass[10pt,journal,a4paper]{IEEEtran}


\usepackage{amsmath}                        % extended math env
\usepackage{amssymb}                        % more math symbols
% \usepackage{balance}                      % balance coloumns on final page
\usepackage{booktabs}                       % nicer tables + hline thickness
% \usepackage{caption}                        % control of caption sizing
\usepackage[usenames,dvipsnames]{color}     % color access

\usepackage{comment}
\let\labelindent\relax
\usepackage[inline]{enumitem}

\usepackage{listings}
\usepackage[inline]{enumitem}
\usepackage[bottom]{footmisc}

\newcommand{\adrien}[1]{\noindent{\color{Green} {\bf \fbox{AG} {\it#1}}}}
% *** CITATION PACKAGES ***
%
\ifCLASSOPTIONcompsoc
  % IEEE Computer Society needs nocompress option
  % requires cite.sty v4.0 or later (November 2003)
  % \usepackage[nocompress]{cite}
\else
  % normal IEEE
  % \usepackage{cite}
\fi


% *** GRAPHICS RELATED PACKAGES ***
%
\ifCLASSINFOpdf
  % \usepackage[pdftex]{graphicx}
  % declare the path(s) where your graphic files are
  % \graphicspath{{../pdf/}{../jpeg/}}
  % and their extensions so you won't have to specify these with
  % every instance of \includegraphics
  % \DeclareGraphicsExtensions{.pdf,.jpeg,.png}
\else
  % or other class option (dvipsone, dvipdf, if not using dvips). graphicx
  % will default to the driver specified in the system graphics.cfg if no
  % driver is specified.
  % \usepackage[dvips]{graphicx}
  % declare the path(s) where your graphic files are
  % \graphicspath{{../eps/}}
  % and their extensions so you won't have to specify these with
  % every instance of \includegraphics
  % \DeclareGraphicsExtensions{.eps}
\fi

\providecommand{\PSforPDF}[1]{#1}


% NOTE: PDF hyperlink and bookmark features are not required in IEEE
%       papers and their use requires extra complexity and work.
% *** IF USING HYPERREF BE SURE AND CHANGE THE EXAMPLE PDF ***
% *** TITLE/SUBJECT/AUTHOR/KEYWORDS INFO BELOW!!           ***
\newcommand\MYhyperrefoptions{bookmarks=true,bookmarksnumbered=true,
pdfpagemode={UseOutlines},plainpages=false,pdfpagelabels=true,
colorlinks=true,linkcolor={black},citecolor={black},pagecolor={black},
urlcolor={black},
pdftitle={How to write an EDIC thesis proposal},%<!CHANGE!
pdfsubject={Typesetting},%<!CHANGE!
pdfauthor={EDIC},%<!CHANGE!
pdfkeywords={EDIC, candidacy exam, EPFL}}%<^!CHANGE!



% correct bad hyphenation here
\hyphenation{op-tical net-works semi-conduc-tor}


\begin{document}

\title{Reviewing kernels \& applications roles}



% The paper headers
\markboth{EDIC Research Proposal}%
{Shell \MakeLowercase{\textit{et al.}}: EDIC Research Proposal}


\IEEEcompsoctitleabstractindextext{%

\begin{abstract}
Modern software applications need to process an ever increasing amount of data, satisfy strict service level objectives, while addressing present-day concerns regarding privacy and security.
Application requirements and architectures have become so complex and heterogeneous that no general-purpose monolithic operating system can fully answer their needs, both in terms of performance and security.
Increasingly often, developers free themselves from the legacy rigid abstractions proposed by the operating system to achieve competitive results.

We believe that the respective roles of the kernel and the application should be re-evaluated to address contemporary performance and security challenges.
We further argue that those challenges can both be addressed with the same fundamental design principles that tend to establish a clear separation between the kernel and the application.
Such a design would allow application-specific management of physical resources, a greater design flexibility,  and to improve overall performances. 
This design would moreover simplify the implementation of new security models that address Cloud deployment confidentiality and integrity concerns, e.g., a mutual distrust between a host and a guest.

We study three papers addressing these challenges and highlight their common design choices:
the exokernel \cite{DBLP:conf/sosp/EnglerKO95} enables application-level management of physical resources, unikernels \cite{DBLP:conf/asplos/MadhavapeddyMRSSGSHC13} leverage the same design to create specialized stand-alone secured and high performance appliances for the Cloud, while Haven \cite{DBLP:journals/tocs/BaumannPH15} provides shielded execution of legacy applications.

Finally, understanding the arguments and solutions presented in these papers, we propose our research goal that pertains to kernel design for modern software applications.
\end{abstract}

\begin{IEEEkeywords}
thesis proposal, candidacy exam write-up, EDIC, EPFL
\end{IEEEkeywords}}

% make the title area
\maketitle

\IEEEdisplaynotcompsoctitleabstractindextext

\IEEEpeerreviewmaketitle



%\section{Introduction}
\section{Introduction}

%1. Applications evolved
%	1.1. Complicated multi-sourced piece of code.
%	1.2. Requirement for performance and security.
%	1.3. Deployed in new ways (e.g., cloud services)
%	
%2. Operating systems are too rigid.
%	2.1. Abstractions have not evolved much in the past decades.
%	2.2. Resource protection as well as management.
%	2.3. Leave little space for application specific logic.
%	2.4. Built in a way that requires them to be trusted.
%	
%3. Objective
%	3.1. Goals is to study designs that allow
%		3.1.1. separate protection from management.
%		3.1.2. Allow application specific management of resources.
%		3.1.3. Symmetric isolation mechanisms (e.g., protect guest from host).
%		3.1.4. Limit attack surface and rely on verification tools.
%	3.2. For that we study 3 different solutions
%		3.2.1 Hardware mechanism with SGX.
%		3.2.2. Kernel design.
%		3.2.3 Software kernel packaged for an application. 

%TODO remove afterwards
%TODO should define that in the paper we say guest for app or vm or anything that runs on top of the host, i.e., the kernel
\subsection{Applications evolved}
Past decades have seen the appearance of new applications.
Available online, spread everywhere, frameworks, libraries.
Multiple source give code that makes current applications ALSO dynamically download code in browser.
This means new security challenges, with fine-grained control over what's happening.
At the same time, explosion of data that circulates, requirement for performance.
Finally, deployed in new ways. Not only running on a personal hardware, but deployed on cloud services, co-located with other applications.
Still requirement for security, isolation, non-trust of the host, performance.\\


%In the past decades, software applications have multiplied, became more complex, and intertwined in new ways.%
%A vast majority of modern applications rely on public API's, external libraries, or software frameworks, hence incorporating code from various, someti%mes untrusted, sources.%
%The web has become the main distribution platform for such software, hence making it impractically hard to verify sources. %TODO reformulate this one.%
%\adrien{Rephrase all of that}The emergence and explosion of such complex systems leads to new security challenges, as verification tools do not allow to ensure a bug free software, applications still need to be executed without being %trusted.
%At the same time, 
%\adrien{Not sure if should keep it or just forget about it}Targeting a framework is much more valuable for attackers as it can potentially affect a large number of applications.
%
%Unavoidable e.g., browsers require to execute javascript, run untrusted web applets etc.
%Browsers need to rely on sandboxing mechanisms to isolate javascript code ran on each page, and download and execute untrusted web applets.
%Another mechanism implemented by certain operating systems consists in maintaining a list of trusted software providers and warn users whenever code from an untrusted source is being executed (Microsoft mitigation mechanism).
%
%Not only did the way applications were developed and distributed changed, but also the way to deploy them.
%Cloud services, for example, host and co-located applications that originate from different sources.
%This new way to deploy applications requires to isolated untrusted applications and prevent them from harming both the underlying host and other co-located applications.
%Still requires to trust the host.
%
%\adrien{Performance???}

\adrien{Need to reformulate everything. Put emphasis on the mutual distrust (don't care about multiple sources) between host and application. Then talk about performance}\\

Cloud computing providers enable even small organizations to deploy web-based services quickly, with low start-up costs, and efficiently adapt the amount of available resources to their current load.
On a machine, the cloud service provider's host divides physical resources among co-located applications from different origins.
While very attractive for their simplicity and adaptability, such services raise important security concerns.

From the service provider's point of view, co-locating mutually distrusted applications requires to
\begin{enumerate*}
	\item isolate applications from each other and
	\item prevent an application from corrupting the host
\end{enumerate*}.
\adrien{Take things from Haven}.
As a result, sandboxing mechanisms became a fundamental building block of today's cloud services.
These mechanisms often follow a classical hierarchical security model with a single-sided isolation mechanism, i.e., privileged trusted code (the host) is protected from the untrusted one (the guest) while retaining access to all the application's data.

From the guest application's point of view, the lack of bidirectional isolation requires to either treat the entire host privileged software stack as a trusted part of the execution, or develop mitigation solutions (e.g., operate on encrypted data) to protect some parts of the user data from the host.

While obvious for Cloud computing, such concerns can be generalized to standard computing systems?

\adrien{Pursue with the need for performance.}

\adrien{At the same time, it is not exactly the same as having access to your own hardware.
The performance is tunable to the extent of the provided service abstractions e.g. bare metal, hybrid or hosted, but with no real control of what is done by the host. Here I should argue more about the performance and the form taken by applications?}

\textit{Summary}: Applications more and more complex, deployed in new ways, require performance, 
\subsection{Operating systems did not}
\adrien{Use the paper sent by Marios for everything that is related to POSIX abstractions and new layered application.
shift of paradigm where applications mostly interface with frameworks that either interact with os or by-pass it to re-implement some low-level features such as network.}

Meanwhile, operating system abstractions did not evolve.
Unix was designed in the seventeens, and is a source of inspiration for the current operating systems.
At the same time, it's the main source of inspiration for POSIX.
Problem is, these abstraction do not take into account new challenges.
That's why some papers break POSIX compatibility, or simply by-pass the kernel.
NEED EXAMPLES (IX can be one for dataplanes, LWC for more flexible memory management).\\

Kernel is responsible for resource protection and management at the same time.
While protection makes sense, i.e., role is to share the resources and ensure each unit some amount,
management seems to be bad, because the kernel is not aware of what the application needs.
A good example is garbage collection (okay I need to find a paper about this, maybe in the exokernel).\\

Finally kernels are all powerful. A compromised kernel can hurt any application.
While denial of service seems unavoidable, what might be a problem is data leakage, corruption, or highjacking of an application.
With more an more co-located applications on cloud services, this is a real problem.
Past efforts were made to protect host from guest, but now we actually want some guarantees the other way around as well.\\

\textit{Summary}: kernel abstractions did not follow the same evolution as applications.
There's a requirement for more liberty in resource management for the application, we want to separate protection from management.
At the same time, we also need to protect guest from host.

\subsection{Objectives}
 In this paper, we want to study existing designs that could better answer today's requirements imposed by the way applications are developped and deployed.
 We want the kernel to focus on protection rather than management.
 The application's knows best how to manage its resources, and should therefore be given access to such resources to do with as they want.
 By leaving the application with most of the management, we de-involve the kernel and hence get a better separation between kern and application, that allows to more easily provide a symmetric isolation/protection mechanisms from host to guest and guest to host.
 We also want to study how modern software tools and techniques can impact kernel design.
 How can we leverage software verification, static analysis and all of that in kernels? 
 C and C++ are the most used languages, we might want a completely type safe language.\\

 \textit{Summary:}This requires modifictions at several levels.
 We study a hardware solution, a kernel one, and a software one.
 We first present the following papers, and then the research proposal.


 


\section{Shielding Applications from an Untrusted Cloud with Haven}

\subsection{Overview}
%Define shieldied application.
%Give general goal and design.

Shielded application.
New problem is that you colocate applications in cloud services and want to be protected against a potentially corrupted host.
Sandbox usually act the other way around.
Rely on hardware mechanisms to do so by using the Intel SGX extension.\\

Whole system is based on the libOS drawbridge.\\

\subsection{The SGX extension}
What is the SGX, what's an enclave, what can you do with it.
Special pages that cannot be accessed or that trigger notification if accessed from outside of the enclave.
How much pages can you put in there?\\

Conclusion of this part must be that it's useful but cannot be enough to provide shielded execution.

\subsection{Design}
The entire "stack" of Haven.
This complicated thing where they have the special untrusted runtime as well.
Things that must be explained:
\begin{itemize}
	\item picoprocess (to protect the host from the guest)
	\item the enclave
	\item the library operating system.
	\item the shield module (how does it verify that host is actually correctly serving a system call?)
	\item untrusted runtime
	\item SGX driver and drawbridge host.
\end{itemize}

\subsection{Evaluation}
Important to see the overhead introduced.
There is a security insentive, but the performance is also important because it is sacrificed here.
Running unmodified binaries is really an advantage.
Disk is still a limitation.
\subsection{Discussion}
%Untrusted time
%VM exits.
%Cannot move as easily as a VM.
%Performance.
%Monitoring is hard when needed.

Still untrusted parts like the time.
Vm exits are a problem.
Cannot move it as easily as a VM, which is a problem since they target cloud deployement.
The performance penalty is pretty high, it improves security but degrades application performance significantely.
Cannot monitor, when on cloud services this is actually required and very important.\\

What we can take away from this paper is actually that applications deployment changed significantely.
Before, we could deploy on our own trusted servers.
Now multiple mutually untrusted applications are colocated on the host and we have this new bidirectional lack of trust between host and guest.
We need to address this issue.
Relying on hardware is an interesting solution because, compared to software, it is less amenable to bugs and hard to attack or circumvent.
The whole libOS thing is also interesting because it shows that by reimplementing the abstractions that regular applications rely on, we can run unmodified applications.

\section{Exokernel: An Operating System Architecture for Application-Level Resource Management}

In traditional monolithic operating systems, applications interact with the kernel via interfaces that hide machine resources behind fixed high-level abstractions such as processes, files, virtual memory, and interprocess communications (IPC).
Unfortunately, as these abstraction cannot be replaced by untrusted user code, their implementations need to be general enough to serve heterogeneous applications.
Consequently, applications that do not need a certain features still incur its associated overhead.
More generally, the lack of flexibility imposed by fixed kernel abstractions impedes applications performance, flexibility, and functionality, by preventing domain-specific optimizations of resource management, the modification of existing abstractions, and requiring new ones to be inefficiently emulated on top of a rigid high-level kernel interface.

Guided by the \emph{end-to-end} principle[CITE], the authors of the Exokernel paper present an ingenious operating system architecture that provides application-level management of physical resources.
In this new model, the kernel securely exports all hardware resources to untrusted, application-specific library operating system.
Library operating systems are then free to efficiently implement their own system objects and management policies.
In the following sections, we focus on the design principles that are at the heart of the Exokernel architecture and report the evaluation of Aegis, a prototype Exokernel.

\subsection{Design Principles}
The main challenge in designing an Exokernel is to enforce a clean separation between protection and mechanisms[CITE].
The kernel avoids resource management up to the extent required by protection.
It is responsible for securely multiplexing hardware resources and allowing different library operating systems (and applications) to coexist on the same machine.
It must do so with little overhead and without imposing design restrictions that might hurt the application's performance.

The role of the Exokernel is therefore to:
\begin{enumerate*}
	\item \label{expose} expose all hardware resources and track ownership throughout the system,
	\item \label{protect} protect resource usage, and
	\item \label{revoke} revoke access to resources
\end{enumerate*}.
Guided by these design principles, the authors made the design choice to rely on physical names to implement three different mechanisms: secure bindings, visible revocation, and an abort protocol.

The use of physical names in allocation and revocation is a central design choice that has several benefits.
First, physical names allow the kernel interface to be as close as possible to the hardware.
Second, it avoids adding an additional level of indirection, e.g., virtual names, that would introduce overheads and force the kernel to perform a complex management of resources.
Third, it preserves the application's ability to take full advantage of the underlying hardware, e.g., by requesting particular physical resources.

To securely expose resources and track ownership, the Exokernel relies on \emph{secure bindings}.
This protection mechanism binds a specific resource to an application at \emph{bind time}, and installs the entries needed to perform efficient ownership checks at \emph{access time}.
Due to the heterogeneity of the multiplexed resources, the Exokernel relies on different techniques to implement secure bindings:
\begin{enumerate*}
	\item hardware support
	\item software caching, and
	\item downloading application code in the kernel
\end{enumerate*}.

Ideally, specific hardware features should be leveraged whenever possible to implement protection and access time checks.
For example, if a hardware memory TLB is available, the kernel first validates a virtual-to-physical memory mapping, e.g., by performing authorization checks on the physical page, and then installs an entry inside the hardware TLB (bind time), which subsequently performs efficient access time checks.

When hardware support is not possible, or to improve the overall system performance, secure bindings can be implemented as a software TLB, i.e., the kernel can cache bindings.
Following the previous example, the kernel can cache extra virtual to physical mappings inside a large software TLB to speed-up the validation and installation of a new hardware TLB entry.

Finally, downloading code in the kernel allows to perform access checks closer to the application's logic, even when it is not scheduled.
The Exokernel authors rely on this technique to multiplex the network and implement packet filtering.

Visible resource revocation involves the application in the process of reclaiming resources.
While it incurs more latency than the invisible revocation approach taken by traditional operating systems, visible revocation presents two main advantages.
First, it does not deny the application's right to efficient resource management.
Second, it does not require the Exokernel to understand how resources are used and more specifically what parts of the current state should be saved.
Resource revocation is an essential mechanism in any operating system, especially in the presence of scarce resources.
Having an efficient and safe protocol for it is therefore vital.

Faced with an uncooperative library operating system, the Exokernel must be able to break secure bindings and reclaim resources.
After an unsuccessful visible revocation phase, the kernel notifies the library operating system that a hard deadline to comply was set.
If the application still fails to satisfy this requirement, the Exokernel reclaims the resource and informs the application that the secure binding was removed.
Another option would be to kill any non-compliant application.
This alternative, however, would hurt application developers that "\textit{have great difficulty reasoning about hard real-time bounds}"(SIC).
Phrased differently, this abort protocol allows the application to recover from a temporary failure to comply with the Exokernel requirements.
Additionally, the Exokernel provides the application with \emph{repossession vectors} to specify beforehand which resources must be saved during an abort protocol.
The Exokernel also guarantees for each application a set of resources that will not be repossessed during an abort protocol and that can be used by vital parts of the program, e.g., exception handlers.
Finally, an emergency protocol allows to swap out an application, i.e., to repossess all resources without "killing" the application.

\subsection{Aegis Performance}

%Aegis
%CPU
%Exception
%Protected control transfer.

%ExOS
%IPC
%virtual memory
%remote communication. 
In this section we report and review the original evaluation of Aegis, a prototype exokernel, and ExOS, a library operating system.
Since the paper was published in 1995 and Aegis evaluated on the hardware available back then, we ignore the measured numerical values and focus on the exokernel performance compared to Ultrix, a traditional monolithic kernel.

Aegis multiplexes the CPU by dividing it into time slices allocated to the different applications.
Timer interrupts are delivered to the application, which in turn performs the context-switch.
To achieve fairness, applications that do not respect such interrupts are forced to give up future slices.
This scheduling mechanism is both flexible and performant.
The cost of a standard procedure call is similar to that of Ultrix.
For system calls, the authors report that Ultrix's \lstinline{getpid} (one of the fastest syscalls) is an order of magnitude slower than Aegis' slowest one.

In accord with the exokernel's design principles, Aegis dispatches all exceptions to applications.
After handling an exception, the application is not required to re-enter the kernel before resuming its execution.
As a result, Aegis's exception mechanism is more than five times faster than the state-of-the-art fastest implementation at that time \cite{DBLP:conf/asplos/ThekkathL94}.

Aegis provides primitives for synchronous and asynchronous protected control transfer, a mechanism that can be relied upon to implement efficient IPC abstractions.
Compared with the then fastest published implementation \cite{DBLP:conf/sosp/Liedtke93}, Aegis mechanism is 6.6 times faster.
This is mostly due to the succinct (30 instructions) implementation of the mechanism in the exokernel.

ExOS is a prototype library operating system developed to run on top of Aegis.
ExOS implements an IPC abstraction on top of Aegis' protected control transfer that is more than an order of magnitude faster than Ultrix equivalent implementation.
A simple pipe implementation is 10 times faster than in Ultrix, while a LRPC is 40 to 60 times faster.
This discrepancy is due to the fact that Ultrix needs to emulate new abstractions, such as LRPC, on top of existing ones, e.g., pipes or signals.

The authors then proceed with a comparison between the unoptimized implementation of virtual memory management in ExOS to the one provided in Ultrix.
First, a simple benchmark allows to show that a standard implementation that does not leverage the exokernel features and Ultrix perform indistinguishably.
A set of microbenchmarks shows that ExOS performs reasonably, i.e., 1.1 to 1.6 times slower, compared to Ultrix on tests that change access rights to contiguous address ranges.
At the same time, ExOS outperforms Ultrix on page-protection traps and more heterogeneous benchmarks by being at least an order of magnitude faster.

\subsection{Discussion}
Good because same design choices than Haven.
In one case it's for security, in the other for performance.
So actually possible to reconcile the two of them.
The main issue now, considering cloud, is how to actually migrate these things?
Also extra advantage can be taken from the fact that the application is one stack (with the libOS).
Authors focused on the fact that your app can be built against a set of default libraries.
But we can do a lot more than that. Now that it forms a whole, we can actually apply PL techniques to it.
That's gonna be the next part.
\adrien{
Cite Sel4 somewhere in the proposal or next section
Finish with the fact that now since we have one stack we can use PL technique.
Since the exokernel is small, we can verify it as well with other techniques e.g., Sel4
How to deploy? Does not solve Haven previous problem}





\section{Unikernels: Library Operating Systems for the Cloud}
%vm pack operating system + application.
%often dedicated to one purpose.
%no specialization
%glue code is used to run different services,
%overhead of everything that is not needed + security issue.

Traditionally, applications are deployed in the Cloud as virtual machines (VM).
A VM packs a complete operating system, such as Linux or Windows, that allows to run unmodified application processes.
Typically, a VM run in the Cloud can be viewed as a single-purpose appliance, i.e., it consists of a guest operating system running a main application, e.g., a database or web server, that relies on smaller services running in parallel.
Due to the lack of standards for application configuration, VMs often run custom scripts as glue code to initialize the execution environment.

Although destined to execute a single main application, Cloud VMs are insufficiently specialized.
They contain an entire operating system image, distributed with some default services.
Apart from unnecessarily increasing the disk image size for this single-purpose appliance, hence requiring more resources to deploy and run the VM, the presence of unwanted default services increases the potential attack surface exposed to a malicious entity and threatens the integrity of the VM.

The Unikernel paper[CITE] presents a new model to deploy applications in the Cloud that addresses the above-mentioned issues by radically specializing the Cloud application's image.
\emph{Unikernels} are small sized sealed single-purpose appliances entirely written in a high-level programming language, that can easily be configured and deployed over cloud platforms.


\subsection{Design Principles}


% 1. Unikernel libOS revisited.
% 	Different than the exokernel, runs on top of a hypervisor, such as Xen.
% 	Why abstraction from hardware is good? Because don't have to write the drivers and can run on the Cloud.
% 2. Unikernels go further than previous libOSes by fully specializing the entire user stack. The libOS and the application are compiled together to yield a standalone kernel to schedule on top of the hypervisor.
% 	Leveraging high-level PL allows to have static type-safety, and use PL tools such as software verification, dead-code elimination.
% 3. Configuration and Deployment
% 	static configuration.

The Unikernel revisists the idea of library operating systems (libOS) by applying it to Cloud platforms.
A unikernel is a standalone kernel that encapsulate a single-purpose application destined to be deployed over the Cloud.
As such, unikernels execute on top of a VM hypervisor, e.g., Xen[CITE].
The advantages of targeting a virtualization platform instead of a custom kernel are twofold.
First, the hypervisor provides a fixed virtual hardware abstraction that alleviates the need to take into account heterogeneous hardware devices.
The hypervisor provides hardware drivers for a large number of devices and abstracts them away behind a fixed high-level interface.
Second, the hypervisor virtual abstraction facilitates both vertical and horizontal dynamic scaling.

% Agressive specialization.
% Via aggressive specialization, achieve:
% 1. Small size compact images.
% 2. Configuration work is taken care of at compile time.

Unikernels achieve aggressive specialization by relying on a single high-level programming language.
System libraries and the application are implemented in a common, statically typed, programming language, and compiled as a single specialized unikernel.
The compiler performs static analysis on the entire user stack and optimizes the whole system at once.
This enables to:
\begin{enumerate*}
	\item perform most of the appliance static configuration at compile time,
	\item reduce the final image's size, and
	\item improve the application's security
\end{enumerate*}, all of which without sacrificing performance.

Static configuration of a unikernel appliance happens at compile time.
As mentioned previously, standard VMs incur a non-negligible deployment overhead.
Custom scripts are executed to configure and start all services needed by the application, e.g., a database for a web server, and glue together independent packages or applications.
Unikernels, on the other hand, integrate all dependencies as libraries, compiled within the application itself.
Build parameters passed to the compilation configuration enable static configuration, while dynamic configuration is achieved via specific function calls to the library.

Unikernels are highly specialized and compact binaries.
Since all the libraries are linked and compiled against the application itself, static analysis techniques such as dead-code elimination (DCE) can be applied to reduce the final image's size.
The authors claim that in most cases, the final binary's size is on the order of kilobytes, and hence easily deployable across the Internet.
Furthermore, in the context of Cloud platforms, where resources are rented, small size binaries consume less resources to host the program's code and data, and hence minimize expenses.

%Dead code removed and reduce attack surface.
%executable excludes writable, limits the attack.
%Can do ASLR.
%Type safety enforced,
%Static analysis tools and verification ones can be used to verify the entire stack.
%What inside the security. Single address space, type safe allows to control flow.
By eschewing backward compatibility and requiring the entire stack to be written in a single high-level typed programming language, unikernels can leverage type-safety and static analysis tools to improve the application's security.
First, DCE reduces the attack surface exposed to a remote attacker system-wide.

\subsection{Evaluation}
\subsection{Discussion}
%\subsection{Overview}
%%TO SAY
%%Keywords: Specialization (of the kernel itself)
%%1. Goal is to have secured, small sized, efficient applications easy to deploy in cloud.
%%2. The application is the kernel. Strip down and seal the app to only what is needed.
%%3. Relies on PL and compiler techniques, e.g., static analysis, static type-safety etc.
%%4. Single address space !!! have to put that somewhere.
%\adrien{When talking about Mirage, say single process for a dedicated virtual CPU. domainpoll to block on event.}
%
%This should be a high level overview of the paper.
%
%\textit{Unikernels} are defined as small sized sealed single-purpose appliances that can easily be deployed over cloud services.
%More specifically, a unikernel encapsulate an application's logic as well as the required system libraries and language runtime.
%However, unlike regular general-purpose VMs, unused functionalities are stripped-away at compile-time therefore yielding a small %sized bootable image while reducing the potential attack surface exposed by said image.\\%
%
%Describe the overall architecture and design.\\
%
%\subsection{Specialization through a unified common language.}
%The application is the OS.
%This goes further than before.\\
%
%Specialising benefits are smaller deployable things.
%A common lower level interface that relies on the supervisor.
%Smaller attack surface.
%Reconfiguration require re-compilation so hard to modify behaviour?\\
%
%Single high level language means that we can leverage modern techniques, e.g. type safety, static analysis, compiler %optimizations to boost performance.%
%At the same time, single address sp%ace is enough because we have the functionalities from the language that act as ba%rriers.
%
%\subsection{Evaluation}
%TODO: redac this\\
%
%Main goals and claims: since supposed to be deployed on the cloud, both size and boot time are relevant.
%This is evaluated qualitatively by comparing LOCs between unikernels and VMs, and by measuring boot time required when deploying %the appliances.%
%Result is that %can boot in less than 50 ms (how does it compare to Linxu?)\\%
%
%Threading is efficient because no kernel and user space separation, hence avoid the mode switch overheads.\\
%
%Network and Storage type safety does not introduce significant delays.
%Thanks to single address space, 0-copy for receive so faster.
%Transmission is, however, slower because requires extra cpu processing (high level language).\\
%
%Application is DNS.
%
%\subsection{Discussion}
%
%Comparison with exokernel: different approach regarding the abstraction level.
%Still trusts the hypervisor, could we use the SGX to protect it somehow?
%Argument that hypervisor enables to get compatibility with drivers is good, but still need to reimplement all the system %libraries...%
%Still suffer%s from the same problems as any LibOS, you need to reimplement the libraries.%
%Other proble%m is that you limit the flexibility by requiring them to be implemented in OCaml.%
%Compatibilit%y with other applications is ensured by having a network communication between th%e appliance and a VM that ru%ns the %service.%
%Not good% because shared memory etc. is not really straight forward.\\
%
%What is better than exokernel? 
%The fact that we can rely on high-level pl tools like verification, dead-code elimination etc.
%Packageable (which is not really the case for an app in exokernel and not a considered concern).\\
%
%What is not as good?
%Everything needs to be implemented in OCaml, because we have the runtime environment and all of that.
%In exokernel, you have an interface, and it could be enough to somehow communicate with it.\\
%
%What is bad as well?
%Can you actually snapshot an appliance and migrate it?
%Check in the paper if this is the case because I'm really not sure.
%

\section{Research Proposal}

While Moore's law is coming to an end, the volume and complexity of data that we need to process keep increasing.
Modern hardware setups harvest the cumulative power of dozens of machines to achieve their computations in a timely fashion.
New hardware appliances are leveraged to off-load parts of the computation and decrease the overall processing time: GPGPUs allow parallel processing, FPGAs pre-processing, and new types of processing units appear, e.g., GAP[CITE] and TPU[CITE].

Meanwhile, the monolithic architecture of the most popular kernels is becoming more and more of a burden.
Based on rigid abstractions devised 50 years ago, it often fails to answer today's challenges in terms of performance and security.
Researchers and developers are forced to rely on kernel bypasses[CITE], virtualization, or software frameworks, to achieve higher throughput, greater flexibility, or improved security.

Operating systems need to adapt quickly to the modern world, and this requires to study alternate designs.
Due to the heterogeneity of requirements and the fast pace at which technology evolves, no general-purpose monolithic system can pretend to provide satisfiable performance for all types of workloads.
Inspired by the Exokernel, systems should therefore focus on securely exposing the available resources and avoid management as much as possible.

My research proposal aims at the study of modern systems, designed to answer present day challenges, i.e.,
\begin{enumerate*}
	\item achieve high performance to allow applications to meet service level objectives,
	\item consider the shift in paradigms such as increasingly parallel and complex hardware setups, and
	\item answer present security requirements, such as mutual distrusts between a system and an application running on top of it.
\end{enumerate*}
To do so, we need to explore the design space between a low level interface, as in the Exokernel, and virtualized environments, as for unikernels.
The ideal solution would be a hybrid design that strikes the sweet spot between exposing raw hardware resources and providing a stable and portable abstraction.

This design would act as a foundation for further exploration of new solutions on higher-level components. 
More precisely, new abstractions could easily be created to leverage the latest hardware features, such as Intel SGX, and create modern execution environments that answer present days security requirements.

As shown in the Unikernel paper, operating systems can benefit from advanced programming languages features, such as memory safety, type safety, intrinsic support for concurrency, or ownership tracking.
Several papers[CITE] already explored how type-safe memory managed languages can be used in the operating system.
With the rise in popularity of new paradigms and models, exported by languages like Rust and its ownership mechanism, or Go and its communicating-sequential-processes (CSP) style for concurrent programming, the space of possible designs and solutions has yet to be explored.

%\section{conclusion}
Applications have evolved, both in the way they are written, and the way they are executed and deployed.
Need new solutions and to redefine the role of operating systems.
Library operating systems proposed a design that can be modified to both enable management of resources closer to the application's logic, as well as including new features that allow mutual distrust between the host and the guest in the case of applications deployed over the cloud.
More than that, sandboxing, dataplanes, user-level implementation of system libraries, all of them can be rendered more efficient by simply providing a common base kernel that exposes flexible abstractions.
We know that the problem is the huge reimplementation of both libs and drivers, as well as extending the amount of untrusted code (i.e., the libraries in the OS themselves).
Taking a step by step approach as done with projects based on Dune is a first thing.
Can the same efforts done by Microsoft can be done for Linux? 
Sandboxes are going to become more and more complicated if the os does not become simpler.
More and more kernel bypass is also to be expected.
Virtualization and kernel modules allows to do just this, but still incure some overhead.

\begin{thebibliography}{1}

\bibitem{mittelbach}
F.~Mittelbach, M.~Goossens, J.~Braams, D.~Carlisle, and C.~Rowley, 
\emph{The {\LaTeX}} Companion, 2nd~ed.\hskip 1em plus
0.5em minus 0.4em\relax Addison-Wesley Professional, 2004.

\end{thebibliography}

% biography section
% 
% If you have an EPS/PDF photo (graphicx package needed) extra braces are
% needed around the contents of the optional argument to biography to prevent
% the LaTeX parser from getting confused when it sees the complicated
% \includegraphics command within an optional argument. (You could create
% your own custom macro containing the \includegraphics command to make things
% simpler here.)
%\begin{biography}[{\includegraphics[width=1in,height=1.25in,clip,keepaspectratio]{mshell}}]{Michael Shell}
% or if you just want to reserve a space for a photo:

%\begin{IEEEbiography}{Michael Shell}
%Biography text here.
%\end{IEEEbiography}

% if you will not have a photo at all:
%\begin{IEEEbiographynophoto}{John Doe}
%Biography text here.
%\end{IEEEbiographynophoto}

% insert where needed to balance the two columns on the last page with
% biographies
%\newpage

%\begin{IEEEbiographynophoto}{Jane Doe}
%Biography text here.
%\end{IEEEbiographynophoto}

% You can push biographies down or up by placing
% a \vfill before or after them. The appropriate
% use of \vfill depends on what kind of text is
% on the last page and whether or not the columns
% are being equalized.

%\vfill

% Can be used to pull up biographies so that the bottom of the last one
% is flush with the other column.
%\enlargethispage{-5in}



% that's all folks
\end{document}


