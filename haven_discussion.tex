%Tries to answer present day problems with an approach that relies on hardware support.
%Seems legit.
%Design principles at the heart of haven or policy mechanism separation, bi-directional mistrust,
%involving the host as little as possible.
%
%But performance issues,
%not a complete system since still requires to trust the host for system time for example.
%Question is, adopting similar design at the heart of the security aspect, can we do better.
Haven addresses the modern confidentiality and integrity concern that sprung from the growth of Cloud services.
In the context of rented resources, users would like to protect the confidentiality and integrity of their data when executing their applications on machines controlled by an untrusted entity.
Haven leverages hardware mechanisms provided by Intel SGX and well-defined design principles to provide shielded execution of legacy binaries.
Haven enforces policies and management separation to disentangle the guest from the host, obtain a cleaner separation between the mutually distrustful domains, and more easily monitor their interactions.
At a high-level, the host is reduced to exposing physical resources, while the guest manages virtual ones.

Unfortunately, as we have seen in the evaluation section, Haven sacrifices performance for confidentiality and integrity.
Parts of the overheads are directly imputable to the hardware mechanisms implemented by Intel SGX and cannot be avoided.
In particular, we observed that memory operations on the EPC are non-negligibly slow.
At the same, Haven's stack itself significantly decreases the application's performance compared to alternate deployments.

The open question is to determine if performance and confidentiality/integrity can be reconciled under a common set of design principles.
In other words, we wonder if the design choices, at the heart of Haven's design, that enforce the isolation between host and guest, preclude performance.
The next paper suggests that it is quite the opposite.