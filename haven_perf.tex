The authors were provided with a functional SGX emulator to implement Haven.
Unfortunately, such an emulator cannot be relied on to measure, or even approximate, the system's performance.
As a result, a second version of Haven that introduces synthetic delays for SGX instructions was used instead.
The performance evaluation hence consists in two steps:
\begin{enumerate*}
	\item compare the performance of Haven with other deployments, where SGX instructions have a null cost and enclave crossings only incur a TLB flush, and
	\item vary the synthetic delays for SGX main features, i.e., enclave crossings, dynamic memory allocation, and accesses to the EPC
\end{enumerate*}.
The first step of the evaluation allows to measure overheads that are purely due to Haven's implementation.
The second step models the performance impact of the SGX hardware.
By cross-referencing Haven's reported results with a recent performance evaluation of SGX \cite{DBLP:conf/IEEEwisa/ZhaoSTZX16}, we are able to get a better understanding of Haven's actual performance.

The first evaluation relies on two real-world applications, namely, a TPC-E benchmark running on a Microsoft SQL server, and an Apache HTTP server benchmark.
Haven's performance is compared with the one achieved by running the same experiments natively, in a hyper-V VM, and in Drawbridge.
As a general result, Haven's throughput is smaller than in all other configurations for both applications.
On the database benchmark, it incurs a 13\% slowdown compared to Drawbridge, which itself is comparable to the VM deployment.
On the HTTP server, Haven exhibits 40\% and 55\% slowdowns compared to Drawbridge and the VM solutions.
We note however that Haven running with a dedicated filesystem outperforms Drawbridge, due to the read-intensive nature of the workload that accesses cached enclave pages.

The second step of the evaluation introduces synthetic delays for enclave crossings, dynamic memory allocation, and accesses to the EPC.
Varying the access latency for EPC pages is done by changing the DRAM clock rate.
Slowing down the memory by a third introduced a 21\% slowdown for the database application, and a 7\% one for the HTTP server.
According to the SGX performance evaluation in \cite{DBLP:conf/IEEEwisa/ZhaoSTZX16}, memory operations inside the enclave are more than three times slower than outside of it.
It therefore seems that Haven's authors largely underestimated the cost of accessing the EPC.

The authors note that while enclave crossings and dynamic allocation of memory both impact the HTTP server application's throughput, the database application seems to be sensitive only to enclave crossings.
By plugging-in the SGX performance evaluation result from \cite{DBLP:conf/IEEEwisa/ZhaoSTZX16}, i.e., 7 kcycles per crossing, we see that enclave crossings induce a 20\% extra slowdown in throughput compared to the previously measured database experiment.
For the HTTP server, the actual slowdown due to enclave crossings is of the order of 1\%.

More generally, Haven's authors predicted a 31\% to 54\% slowdown in performance compared to a virtual machine solution.
As mentioned above, memory operations on EPC pages were largely underestimated.
A thorough experiment on real hardware is needed to better understand how Haven really compares to VM based solutions.
We also note that, while targeting Cloud services, Haven does not provide the same flexibility as VMs in terms of deployment and migration of applications.
