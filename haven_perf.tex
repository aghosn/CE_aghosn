The authors were provided with a functional SGX emulator to implement Haven.
Unfortunately, such an emulator cannot be relied on to measure, or even approximate, the system performance.
As a result, the authors implemented a second version of Haven that introduces synthetic delays for SGX instructions.
The performance evaluation hence consists in two steps:
\begin{enumerate*}
	\item Compare the performance of Haven with other deployments, where SGX instructions have a null cost and enclave crossings only incur a TLB flush, and
	\item vary the synthetic delays for SGX main features, i.e., enclave crossings, dynamic memory allocation, and accesses to the EPC
\end{enumerate*}.

The first step of the evaluation allows to measure overheads that are purely due to Haven's implementation.
The second step models the performance impact of the SGX hardware.
By cross-referencing Haven's reported results with a recent performance evaluation of SGX[CITE], we are able to get a better understanding of Haven actual performance.

The first evaluation relies on two real-world applications, namely, TPC-E benchmark running on a Microsoft SQL server, and an Apache HTTP server benchmark.
Haven's performance is compared with the throughputs achieved by running the same experiments natively, in a hyper-V VM, and in Drawbridge.
As a general result, Haven's throughput is inferior to all other configuration in both applications.
On the database benchmark, it incurs a 13\% slowdown compared to Drawbridge.
On the HTTP server, the same comparison yields a 40\% slowdown.
We note however that Haven running with a dedicated filesystem outperforms Drawbridge, due to the read-intensive nature of the workload that access cached pages inside the enclave.

The second step of the evaluation introduces synthetic delays for enclave crossings, dynamic memory allocation, and accesses to the EPC.
First, varying the delay to access EPC pages is done by changing the DRAM clock rate.
Slowing down the memory by a third introduced a 21\% slowdown for the database application, and a 7\% one for the HTTP server.
Second, the authors note that while enclave crossings and dynamic allocation of memory both have impacts on the HTTP server application's throughput, the database application seems to be sensitive only to enclave crossings.
By plugging-in the SGX performance evaluation result from[CITE], we see that enclave crossings induce a 20\% extra slowdown in throughput in the previously measured Haven's database experiment (8 kcycles per crossing).
For the HTTP server, the actual slowdown is of the order of 1\% for crossings.

More generally, Haven's author predicted a 31\% to 54\% slowdown in performance compared to a virtual machine solution.
This estimation seems conservatively correct when actual SGX performances are considered.

To get a better understanding of SGX performance, we briefly report the result from[CITE].
Cross enclave calls require 7 kcycles, which is an order of magnitude greater than a standard system call in a Linux operating system.
Memory operations inside an enclave are a little more than three times slower than outside of it, while cross enclave memory copy exhibits a 50\% slowdown in throughput.